\documentclass[12pt]{article}

\usepackage[table]{xcolor}
%  \usepackage{crayola}
\usepackage{amsfonts}
\usepackage{amsmath}
%\usepackage{amsthm}
\usepackage[procnames]{listings}
\usepackage{url}
\usepackage{mybold}
\usepackage{mycal}
\usepackage{mymathbb}

%  \input ../doc/myarticle.sty

\topmargin=-0.5in\oddsidemargin=0pt\topskip=\baselineskip
\headsep=0pt\headheight=0.4in
\textwidth=6.6 true in\textheight=9.0 true in
\parskip = 10pt
\parindent = 0pt


\lstloadlanguages{C}
\lstset{language=C,
float=tbhp,
captionpos=t,
frame=trbl,
abovecaptionskip=0.2em,
belowskip=0.3em,
basicstyle=\footnotesize\ttfamily,
stringstyle=\color{OliveGreen},
commentstyle=\color{red},
identifierstyle=\color{Bittersweet},
basewidth={0.5em},
showstringspaces=false,
framerule=0.8pt,
procnamestyle=\bfseries\color{blue},
emphstyle=\bfseries\color{Cerulean},
procnamekeys={class,extends,interface,implements}
}

\def\bs{{\tt\char'134}} 

\def\red#1{{\color{red} #1}}
\def\blue#1{{\color{blue} #1}}
\def\green#1{{\color{green} #1}}
\def\orange#1{{\color{orange} #1}}

\newcommand\guisec[1]{\vspace{10pt}%
  \noindent\hrulefill\hspace{10pt}{\bf #1}\hspace{10pt}\hrulefill
  \vspace{10pt}\nopagebreak}

\catcode`\@=11
\def\inc#1{\@partswtrue\edef\@partlist{#1}}

% Macros pour inserer des bouts de code (programmes).
% Faire  \code ...  \endcode
{\obeyspaces\gdef {\ }}
\def\setverbatim{\def\par{\leavevmode\endgraf}
            \parskip=0pt\parindent=0pt\obeylines\obeyspaces }
\chardef\other=12
\def\ttverbatim{\setverbatim\tt
       \catcode`\{=\other \catcode`\}=\other \catcode`\_=\other
       \catcode`\^=\other \catcode`\$=\other \catcode`\%=\other
       \catcode`\#=\other \catcode`\&=\other \baselineskip=11pt
       }
    % Reproduit tel quel ce qui est ecrit, en caracteres \tt.
    % On doit faire  \begingroup\ttverbatim   ....  \endgroup
\def\smallttverbatim{\ttverbatim\small\tt}
\def\code {\vfil\vfilneg\vbox\bgroup\ttverbatim}
\def\longcode {\vfil\vfilneg\bgroup\ttverbatim}
\def\smallc {\small\tt\baselineskip=9.5pt}
\def\footc {\footnotesize\tt\baselineskip=9.0pt}
\def\smallcode {\code\smallc}
\let\endcode=\egroup
\let\vcode=\code
\let\endvcode=\egroup

%  Definition d'un module ou d'une classe.
\def\ps@nomark {\def\leftmark{} \def\rightmark{}}
\def\defmodule#1 {\addcontentsline{toc}{subsection}{#1} \markboth{#1}{#1}
   \centerline {\LARGE\bf #1}\bigskip \thispagestyle{nomark}}
\def\defclass#1 {\addcontentsline{toc}{subsection}{#1} \markboth{#1}{#1}
   \centerline {\LARGE\bf #1}\bigskip \thispagestyle{nomark}}
%  Lorsqu'on veut cacher certaines choses a l'usager, faire  \hide ... \endhide
\newif\iffull\fullfalse
\def\hide{\iffull}
\let\endhide=\fi

\def\parup{\nobreak\vskip -2pt\nobreak}

\def\tab{\small\dimen9=\parindent\parindent=0pt%
   \advance\leftskip by 1.5em\parup}
\def\tabb{\small\dimen9=\parindent\parindent=0pt%
   \advance\leftskip by 3.0em\parup}
\def\tabbb{\small\dimen9=\parindent\parindent=0pt%
   \advance\leftskip by 4.5em\parup}
\def\endtab{\vskip 0.01pt\advance\leftskip by -1.5em\normalsize%
   \parindent=\dimen9}
\def\endtabb{\vskip 0.01pt\advance\leftskip by -3.0em\normalsize%
   \parindent=\dimen9}
\def\endtabbb{\vskip 0.01pt\advance\leftskip by -4.5em\normalsize%
   \parindent=\dimen9}

% Pour mettre quelque chose dans une boite double.
\def\boxit#1{\vbox{\hrule height1pt
                   \hbox{\vrule width1pt\kern3pt
                         \vbox{\kern3pt#1\kern3pt
                              }\kern3pt\vrule width1pt
                        }\hrule height1pt }}
\def\boxr#1{\hfil\vbox{\hrule height1pt
                       \hbox{\vrule width1pt\kern3pt
                              \vbox{#1}\hfil
                              \kern3pt\vrule width1pt
                             }\hrule height1pt }}

% Synonymes plus courts pour  \begin{equation}, \begin{eqnarray}, etc.
\def\eq{\equation}  \def\endeq{\endequation}
\def\eqs{\eqnarray} \def\endeqs{\endeqnarray}
\def\eqsn{\begin {eqnarray*}} \def\endeqsn{\end{eqnarray*}}

% Proof avec boite alignee a droite a la fin.
\newenvironment{myproof}{{\em Proof.}}{\hspace*{\fill}$\Box$}
% \newenvironment{proof}{{\em Proof.}}{\hspace*{\fill}$\Box$}

% Macros (avec switch) pour faire apparaitre les noms des labels dans les eqs.
% Utiliser \eqlabel au lieu de \label.  Doit laisser un espace apres le }
% Pour que les etiquettes apparaissent, faire \seeeqlabelstrue  au debut.
\newif\ifseeeqlabels\seeeqlabelsfalse
\newbox\eqlab \setbox\eqlab=\hbox {}
\def\eqlabel#1 {\global\setbox\eqlab=\hbox
   {\ifseeeqlabels {\rm (#1)} \else {} \fi } \label{#1} }
\def\@eqnnum {{\rm \box\eqlab \setbox\eqlab=\hbox {} (\theequation)}}

% Comme ci-haut pour \eqlabel, mais pour les noms des autres labels
% (Theoremes, Propositions, etc.).
% Utiliser \vislabel au lieu de \label.
% Pour que les etiquettes apparaissent, faire \seevislabelstrue  au debut.
\newif\ifseevislabels\seevislabelsfalse
\def\vislabel#1 {\ifseevislabels {\ \em (#1).\ } \else {} \fi \label{#1} }

% Pour mettre des remarques temporaires.
\newif\ifREM\REMfalse
\def\REM#1 {\ifREM  \begin{quote} \small\em #1 \end{quote} \else {\null} \fi }

% Pour avoir "running head" et no. de page en haut de page, faire  \mytwoheads
% Si on veut la date en haut de chaque page, on fait aussi  \dateheadtrue
\newif\ifdatehead\dateheadfalse
\def\mytwoheads {\pagestyle{headings}
  \topmargin=-0.4in\headheight=0.2in\headsep=0.4in
  \oddsidemargin=0.2in\evensidemargin=0in
  \def\@evenhead {{\large\bf\thepage}\quad\leftmark\hfil
    \ifdatehead\small\it\today\fi}%         Left heading
  \def\@oddhead {\ifdatehead{\small\it\kern-1em\today}\fi\hfil
    \rightmark\quad\large\bf\thepage}}%     Right heading

\catcode`\@=12


\newif\ifnotes\notestrue
%\notesfalse             %%  Uncomment this line to hide footnotes.  <----
\def\boxnote#1#2{\ifnotes\fbox{\footnote{\ }}\ \footnotetext{ From #1: #2}\fi}
\def\pierre#1{\boxnote{Pierre}{\color{red}#1}}
\def\mpierre#1{{\color{red} #1}}
\def\mp#1{{\color{red} #1}}
\def\hpierre#1{}
\def\hrichard#1{}
\def\richard#1 {\fbox {\footnote {\ }}\ \footnotetext { From Richard: #1}}

\def\simon#1{\boxnote{Simon}{\color{cyan}#1}}   
\newcommand{\msimon}[1]{{\color{cyan}#1}}        
\def\hsimon#1{}                                  

\def\perhaps#1{{\color{gray} #1}}

\newif\ifdetailed\detailedfalse
\newenvironment{detailed}{\protect\ifdetailed }{\protect\fi }


%%%%%%%%%%%%%%%%%%%%%%%%%%%%%%%%%%%%%%%%%%
\begin{document}

%  \begin{titlepage}

\null 
\begin {flushright} \it Last update: \today \end {flushright}

\vfill
\begin{center}
 {\Large\bf Unit-tests} \\ \ \\
 {\large\bf Unit Testing for the Modules of Mylib-C}\\
\vfill
 {\bf Pierre L'Ecuyer, Robin Legault and Simon-Olivier Laperrière} \\ \ \\
\medskip
D\'epartement d'Informatique et de Recherche Op\'erationnelle \\
Universit\'e de Montr\'eal \\
\end{center}
\vfill

This document describes a collection of unit tests for the modules of \emph{Mylib-C}. It provides
a set of functions to test the utility functions of the library.
For each module defined in {\bf mylib}, a corresponding module in {\bf unit-tests} offers functions that test out the functions of said module. The module {\tt unit} contains functions to test the whole library at the same time.

\vfill
\end{titlepage}

\begin{titlepage}

\null 
\begin {flushright} \it Last update: \today \end {flushright}

\vfill
\begin{center}
 {\Large\bf Mylib-C Unit Tests} \\
%  {\large\bf Unit Testing for the Modules of Mylib-C}\\
\vfill
 {\bf Simon-Olivier Laperri\`ere and Robin Legault} \\ \ \\
\medskip
D\'epartement d'Informatique et de Recherche Op\'erationnelle \\
Universit\'e de Montr\'eal \\
\end{center}
\vfill

This document describes a collection of unit tests for the modules of \emph{Mylib-C}. 
% It provides a set of functions to test the utility functions of the library.
For each module defined in {\tt mylib}, a corresponding module in {\tt unit-tests} provides
functions that test the functions of that module. 
The module {\tt mylib\_utest} contains functions to test the whole library by a single function call.

\vfill
\end{titlepage}

%%%%%%%%%%%%%%%%%%%%%%%%%%%%%%%%%%%%%%%%%%
\pagenumbering{roman}

%  \section*{Copyright}
 
% This file is part of Mylib-C.

Copyright \copyright {} 2002--2020 by Universit\'e de Montr\'eal.\\
  Web address:   \url{http://www.iro.umontreal.ca/~lecuyer/} \\
  All rights reserved.

Licensed under the Apache License, Version 2.0 (the "License"); 
you can use this software only in compliance with the License. 
You can obtain a copy of the License at

http://www.apache.org/licenses/LICENSE-2.0

Unless required by applicable law or agreed to in writing, software distributed under the License is distributed on an "AS IS" BASIS, WITHOUT WARRANTIES OR CONDITIONS OF ANY KIND, either express or implied. See the License for the specific language governing permissions and limitations under the License.

In scientific publications that used this software, a reference to it and to the user guide would be appreciated.

See the User's Guide for the list of contributors.
\section*{Copyright}
 
% This file is part of Mylib-C.

Copyright \copyright {} 2002--2020 by Universit\'e de Montr\'eal.
%  Web address:   \url{http://www.iro.umontreal.ca/~lecuyer/} \\
  All rights reserved.

Licensed under the Apache License, Version 2.0 (the "License"); 
you can use this software only in compliance with the License. 
You can obtain a copy of the License at

http://www.apache.org/licenses/LICENSE-2.0

Unless required by applicable law or agreed to in writing, software distributed under the License is distributed on an "AS IS" BASIS, WITHOUT WARRANTIES OR CONDITIONS OF ANY KIND, either express or implied. See the License for the specific language governing permissions and limitations under the License.

\clearpage
%  \addcontentsline{toc}{subsection}{Copyright}
\tableofcontents
\clearpage

\pagenumbering{arabic}

\iffalse  %%%%%
\defmodule {mylib\_utest}

This module defines a few macros and functions used to shorten the code of the unit tests.
%  These unit tests ensure that the utility functions defined in the library are correct. 
It also offers functions that execute all the unit tests of the library in one call.

%%%%%%%%%%%%%
\bigskip
\hrule
\code
\hide
#ifndef MYLIB_UTEST_H
#define MYLIB_UTEST_H
\endhide

#include "gdef.h"
#include <stdio.h>
#include <string.h>

#define assert_double(x,y) util_Assert(util_nearEqualDefault(x,y), "Failure in double comparison");
\endcode
 \tab  This macro compares double values {\tt x} and {\tt y}.
  It returns an error message if these values do not correspond.
 \endtab
\code

#define assert_int64_t(n1, n2) util_Assert(n1 == n2, "Failure in 64-bit comparison");
\endcode
 \tab  This macro compares 64-bit integer values {\tt n1} and {\tt n2}.
  It returns an error message if these values do not correspond.
 \endtab
\code

#define assert_uint64_t(n1, n2) util_Assert(n1 == n2, "Failure in unsigned 64-bit comparison");
\endcode
 \tab  This macro compares unsigned 64-bit integer values {\tt n1} and {\tt n2}.
  It returns an error message if these values do not correspond.
 \endtab
\code

#define assert_int(n1, n2) util_Assert(n1 == n2, "Failure in int comparison");
\endcode
 \tab  This macro compares integer values {\tt n1} and {\tt n2}.
  It returns an error message if these values do not correspond.
 \endtab
\code

#define assert_str(str1, str2) util_Assert(strcmp(str1,str2) == 0, "Failure in string comparison");
\endcode
 \tab  This macro compares strings {\tt str1} and {\tt str2}.
  It returns an error message if these two strings are not the same.
 \endtab
\code

void mylib_utest_testall();
\endcode
 \tab  This function runs all the unit tests of the library. 
  It can be executed after the first compilation of the library to ensure that the installation 
	process has worked correctly.
 \endtab
\code

\hide
#endif
\endhide
\endcode

\defmodule {num\_utest}

This module contains unit tests for the functions implemented in {\tt num}. 

%%%%%%%%%%%%%
\bigskip
\hrule
\code
\hide
#ifndef NUMUTEST_H
#define NUMUTEST_H
\endhide
\endcode

\code
void num_utest_Round64();
\endcode
 \tab  Unit testing of the function {\tt num\_Round64}.
 \endtab
\code

void num_utest_RoundD();
\endcode
 \tab  Unit testing of the function {\tt num\_RoundD}.
 \endtab
\code

void num_utest_IsNumber();
\endcode
 \tab  Unit testing of the function {\tt num\_IsNumber}.
 \endtab
\code

void num_utest_IntToStrBase();
\endcode
 \tab  Unit testing of the function {\tt num\_IntToStrBase}.
 \endtab
\code

void num_utest_MultMod();
\endcode
 \tab  Unit testing of the function {\tt num\_MultMod}.
 \endtab
\code

void num_utest_MultModDouble();
\endcode
 \tab  Unit testing of the function {\tt num\_MultModDouble}.
 \endtab
\code

void num_utest_MultModDirect();
\endcode
 \tab  Unit testing of the function {\tt num\_MultModDirect}.
 \endtab
\code

void num_utest_InvEuclid();
\endcode
 \tab  Unit testing of the function {\tt num\_InvEuclid}.
 \endtab
\code

void num_utest_InvEuclid32();
\endcode
 \tab  Unit testing of the function {\tt num\_InvEuclid32}.
 \endtab
\code

void num_utest_InvExpon();
\endcode
 \tab  Unit testing of the function {\tt num\_InvExpon}.
 \endtab
\code

void num_utest_InvExpon32();
\endcode
 \tab  Unit testing of the function {\tt num\_InvExpon32}.
 \endtab
\code

void num_utest_gcd();
\endcode
 \tab  Unit testing of the function {\tt num\_gcd}.
 \endtab
\code

void num_utest_gcd32();
\endcode
 \tab  Unit testing of the function {\tt num\_gcd32}.
 \endtab
\code

void num_utest_isMersennePrime();
\endcode
 \tab  Unit testing of the function {\tt num\_isMersennePrime}.
 \endtab
\code

void num_utest_all();
\endcode
 \tab  This function executes all the unit tests of {\tt num}.
 \endtab
\code


\hide
#endif
\endhide
\endcode

\defmodule {num2\_utest}

This module contains unit tests for the functions implemented in {\tt num2}. 

%%%%%%%%%%%%%
\bigskip
\hrule
\code
\hide
#ifndef NUM2UTEST_H
#define NUM2UTEST_H
\endhide
\endcode

\code
void num2_utest_LnFactorial();
\endcode
 \tab  Unit testing of the function {\tt num2\_LnFactorial}.
 \endtab
\code

void num2_utest_LnGamma();
\endcode
 \tab  Unit testing of the function {\tt num2\_LnGamma}.
 \endtab
\code

void num2_utest_Combination();
\endcode
 \tab  Unit testing of the function {\tt num2\_Combinaison}.
 \endtab
\code

void num2_utest_CalcMatStirling();
\endcode
 \tab  Unit testing of the function {\tt num2\_CalcMatStirling}.
 \endtab
\code

void num2_utest_VolumeSphere();
\endcode
 \tab  Unit testing of the function {\tt num2\_VolumeSphere}.
 \endtab
\code

void num2_utest_BesselK025();
\endcode
 \tab  Unit testing of the function {\tt num2\_BesselK025}.
 \endtab
\code

void num2_utest_Digamma();
\endcode
 \tab  Unit testing of the function {\tt num2\_Digamma}.
 \endtab
\code

void num2_utest_all();
\endcode
 \tab  This function executes all the unit tests of {\tt num2}.
 \endtab
\code\hide
#endif
\endhide
\endcode

\defmodule {bitset_utest}

Definition/example

%%%%%%%%%%%%%
\bigskip
\hrule
\code
#ifndef BITSETUTEST_H
#define BITSETUTEST_H
\endcode

\code

void bitset_utest_ReverseOrderSimple();
void bitset_utest_ReverseOrder();
void bitset_utest_all();

#endif
\endcode

\defmodule {bitvector\_utest}

This module contains unit tests for the functions implemented in {\tt bitvector}. 

%%%%%%%%%%%%%
\bigskip
\hrule
\code
\hide
#ifndef BITVECTORUTEST_H
#define BITVECTORUTEST_H
\endhide
\endcode

\code
void bitvector_utest_copy();
\endcode
 \tab  Unit testing of the function {\tt bitvector\_copy}.
 \endtab
\code

void bitvector_utest_copyPart();
\endcode
 \tab  Unit testing of the function {\tt bitvector\_copyPart}.
 \endtab
\code

void bitvector_utest_clearVector();
\endcode
 \tab  Unit testing of the function {\tt bitvector\_clearVector}.
 \endtab
\code

void bitvector_utest_clearBit();
\endcode
 \tab  Unit testing of the function {\tt bitvector\_clearBit}.
 \endtab
\code

void bitvector_utest_canonical();
\endcode
 \tab  Unit testing of the function {\tt bitvector\_canonical}.
 \endtab
\code

void bitvector_utest_setAllOnes();
\endcode
 \tab  Unit testing of the function {\tt bitvector\_setAllOnes}.
 \endtab
\code

void bitvector_utest_isZero();
\endcode
 \tab  Unit testing of the function {\tt bitvector\_isZero}.
 \endtab
\code

void bitvector_utest_areEqual();
\endcode
 \tab  Unit testing of the function {\tt bitvector\_areEqual}.
 \endtab
\code

void bitvector_utest_haveCommonBit();
\endcode
 \tab  Unit testing of the function {\tt bitvector\_haveCommonBit}.
 \endtab
\code

void bitvector_utest_xor();
\endcode
 \tab  Unit testing of the function {\tt bitvector\_xor}.
 \endtab
\code

void bitvector_utest_xor3();
\endcode
 \tab  Unit testing of the function {\tt bitvector\_xor3}.
 \endtab
\code

void bitvector_utest_xorSelf();
\endcode
 \tab  Unit testing of the function {\tt bitvector\_xorSelf}.
 \endtab
\code

void bitvector_utest_and();
\endcode
 \tab  Unit testing of the function {\tt bitvector\_and}.
 \endtab
\code

void bitvector_utest_andSelf();
\endcode
 \tab  Unit testing of the function {\tt bitvector\_andSelf}.
 \endtab
\code

void bitvector_utest_andMaskLow();
\endcode
 \tab  Unit testing of the function {\tt bitvector\_andMaskLow}.
 \endtab
\code

void bitvector_utest_andInvMaskLow();
\endcode
 \tab  Unit testing of the function {\tt bitvector\_andInvMaskLow}.
 \endtab
\code

void bitvector_utest_leftShift();
\endcode
 \tab  Unit testing of the function {\tt bitvector\_leftShift}.
 \endtab
\code

void bitvector_utest_rightShift();
\endcode
 \tab  Unit testing of the function {\tt bitvector\_rightShift}.
 \endtab
\code

void bitvector_utest_leftShiftSelf();
\endcode
 \tab  Unit testing of the function {\tt bitvector\_leftShiftSelf}.
 \endtab
\code

void bitvector_utest_rightShiftSelf();
\endcode
 \tab  Unit testing of the function {\tt bitvector\_rightShiftSelf}.
 \endtab
\code

void bitvector_utest_flip();
\endcode
 \tab  Unit testing of the function {\tt bitvector\_flip}.
 \endtab
\code

void bitvector_utest_setMaskLow();
\endcode
 \tab  Unit testing of the function {\tt bitvector\_setMaskLow}.
 \endtab
\code

void bitvector_utest_setInvMaskLow();
\endcode
 \tab  Unit testing of the function {\tt bitvector\_setInvMaskLow}.
 \endtab
\code

void bitvector_utest_all();
\endcode
 \tab  This function executes all the unit tests of {\tt bitvector}.
 \endtab
\code


\hide
#endif
\endhide
\endcode

\defmodule {bitmatrix_utest}

Definition/example

%%%%%%%%%%%%%
\bigskip
\hrule
\code
#ifndef BITMATRIXUTEST_H
#define BITMATRIXUTEST_H
\endcode

\code
void bitmatrix_utest_copypart();
void bitmatrix_utest_copySpecial();
void bitmatrix_utest_transpose();
void bitmatrix_utest_exchangeRows();
void bitmatrix_utest_xorVect();
void bitmatrix_utest_diagonalize();
void bitmatrix_utest_gaussianElimination();
void bitmatrix_utest_specialGaussianElimination();
void bitmatrix_utest_completeElimination();
void bitmatrix_utest_inverse();
void bitmatrix_utest_productByVector();
void bitmatrix_utest_product();
void bitmatrix_utest_power();
void bitmatrix_utest_powerOfTwo();
void bitmatrix_utest_all();

#endif
\endcode

\defmodule {rngstream_utest}

Definition/example

%%%%%%%%%%%%%
\bigskip
\hrule
\code
#ifndef RNGSTREAMUTEST_H
#define RNGSTREAMUTEST_H
\endcode

\code

rngstream_utest_RandU01();
rngstream_utest_RandInt();
rngstream_utest_SetSeed();
rngstream_utest_SetPackageSeed();
rngstream_utest_all();

#endif
\endcode

\fi  %%%%

\clearpage\input mylib_utest.tex
\clearpage\input num_utest.tex
\clearpage\input num2_utest.tex
\clearpage\input bitset_utest.tex
\clearpage\input bitvector_utest.tex
\clearpage\input bitmatrix_utest.tex
\clearpage\input rngstream_utest.tex
\clearpage

% \appendix
%  \bibliographystyle {plain}
%  \bibliography{stat,random,simul,math}
\end{document}