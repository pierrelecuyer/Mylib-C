\defmodule {mylib\_utest}

This module defines a few macros and functions to produce the unit tests and ensure that the utility functions defined in the library are correct. It also offers functions that execute all the unit tests of the library in one call.

%%%%%%%%%%%%%
\bigskip
\hrule
\code
\hide
#ifndef MYLIB_UTEST_H
#define MYLIB_UTEST_H
\endhide

#include "gdef.h"
#include <stdio.h>
#include <string.h>

#define assert_double(x,y) util_Assert(util\_nearEqualRel(x,y), "Failure in double comparison");
\endcode
 \tab  This macro compares double values {\tt x} and {\tt y}. 
  It returns an error message if these values do not correspond.
 \endtab
\code

#define assert_int64_t(n1, n2) util_Assert(n1 == n2, "Failure in 64-bit comparison");
\endcode
 \tab  This macro compares 64-bit integer values {\tt n1} and {\tt n2}. 
  It returns an error message if these values do not correspond.
 \endtab
\code

#define assert_uint64_t(n1, n2) util_Assert(n1 == n2, "Failure in unsigned 64-bit comparison");
\endcode
 \tab  This macro compares unsigned 64-bit integer values {\tt n1} and {\tt n2}. 
  It returns an error message if these values do not correspond.
 \endtab
\code

#define assert_int(n1, n2) util_Assert(n1 == n2, "Failure in int comparison");
\endcode
 \tab  This macro compares integer values {\tt n1} and {\tt n2}. 
  It returns an error message if these values do not correspond.
 \endtab
\code

#define assert_str(str1, str2) util_Assert(compare(strcmp(str1,str2),0), "Failure in string comparison");
\endcode
 \tab  This macro compares strings {\tt str1} and {\tt str2}. 
  It returns an error message if these two strings are not the same.
 \endtab
\code

lebool compare(double x, double x);
\endcode
 \tab  This function allow to compare double values {\tt x} and {\tt y} with acceptable 
   relative error less than $10^{-12}$.
 \pierre{Replace this everywhere by \texttt{util\_nearEqualRel(x,y)} instead.} 
 \endtab
\code

void mylib_utest_testall();
\endcode
 \tab  This function runs all the unit tests of the library. It can be executed after the first compilation of the library to ensure that the installation process has worked without glitch.
 \endtab
\code

\hide
#endif
\endhide
\endcode
