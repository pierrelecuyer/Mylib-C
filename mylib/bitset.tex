\defmodule {bitset}

This module defines sets of bits and useful operations for such sets.
Some of these operations are implemented as macros.
Each bit set is stored as a 64-bit unsigned integer, whose bits are simply interpreted
as indicators of which elements belong to the set.
The bits are numbered from 0 to 63, and
\emph{bit 0 is the least significant bit} of this word,
which means that the bits are numbered from right to left.
\hpierre{Here the bits are numbered from the right, whereas they are
  numbered from the left in \texttt{bitvectors}.}%
%	Are there bitsets like this in other libraries?}
If bit $j$ is 1, then element $j$ is a member of the set, otherwise it is not.
Other operations not described here can also be applied directly
to the \texttt{uint64\_t} variable.


\code\hide
/* bitset.h  for ANSI C */
#ifndef BITSET_H
#define BITSET_H
#include "gdef.h"
\endhide\endcode


%%%%%%%%%%%%%%%%%%%%%%%%%%%%%%%%%%%%%%%%%%
\guisec{Constants}
\code

extern const uint64_t bitset_ONE;
extern const uint64_t bitset_ALLONES;
\endcode
\tab
  The two constants 1 and $2^{64}-1$ (a bitset with all ones).
\endtab
\iffalse  %%%%
\code

extern uint64_t bitset_MASK1[];
\endcode
 \tab {\tt bitset\_MASK1[j]} has only \emph{bit} $j$ set to 1 and all other
  bits set to 0.
  \pierre{Not sure if it is worthwhile to store and retrieve all these constants.
	  See \url{http://c-faq.com/misc/bitsets.html} and
    \url{https://github.com/iplinux/x11proto-trap/blob/master/xtrapbits.h}
		for alternatives.}
 \endtab
\code

extern uint64_t bitset_MASK[];
\endcode
 \tab {\tt bitset\_MASK[j]} has all \emph{the first $j$ bits} set to 1 and all other
  bits set to 0.
 \endtab
\fi  %%%%%

%%%%%%%%%%%%%%%%%%%%%%%%%%%%%%%%%%%%%%%%%%
\guisec{Types}
\code

typedef uint64_t bitset_BitSet;
\endcode
 \tab  A set of bits. The bits are numbered starting from 0 for the least significant bit.
 \endtab


%%%%%%%%%%%%%%%%%%%%%%%%%%%%%%%%%%%%%%%%%%
\guisec{Macros}

\code

#define bitset_Mask1(b) (bitset_ONE << b)
\endcode
 \tab Gives a bit set with only bit $b$ set to 1 and all other bits set to 0.
 \endtab
\code

#define bitset_MaskLow(b) (bitset_ALLONES >> (64-b))
\endcode
 \tab Gives a bit set with all \emph{the lowest (rightmost) $b$ bits} set to 1
  and all other bits set to 0.
 \endtab
\code

#define bitset_SetBit(s, b) ((s) |= (bitset_Mask1(b)));
\endcode
 \tab  Sets bit {\tt b} in set  {\tt s}  to 1.
 \endtab
\code

#define bitset_ClearBit(s, b) ((s) &= ~(bitset_Mask1(b)))
\endcode
 \tab  Sets bit {\tt b} in set  {\tt s}  to 0.
 \endtab
\code

#define bitset_FlipBit(s, b) ((s) ^= (bitset_Mask1(b)))
\endcode
 \tab  Flips bit {\tt b} in set {\tt s};  thus,
        $0 \rightarrow 1$ and $1 \rightarrow 0$.
 \endtab
\code

#define bitset_GetBit(s, b)  ((s) & (bitset_Mask1(b)) ? 1 : 0)
\endcode
 \tab  Returns the value of bit {\tt b} in set {\tt s} (0 or 1).
 \endtab
\code

#define bitset_RotateLeft(s, r)  ((s << r) | (s >> (64 - r)))
\endcode
 \tab  Rotates the bit set {\tt s} by {\tt r} positions to the left.
 \endtab
\code

#define bitset_RotateRight(s, r)  ((s >> r) | (s << (64 - r)))
\endcode
 \tab  Rotates the bit set {\tt s} by {\tt r} positions to the right.
\endtab

\noindent
{\tt\#define bitset\_RotateLeftLocal(s, b, r)}

\hide\code
#define bitset_RotateLeftLocal(s, b, r) do { \
   uint64_t temp = (s) >> ((b) - (r)); \
   (s) <<= (r);   (s) |= temp;   (s) &= bitset_MaskLow(b); \
   } while (0)
\endcode\endhide
\tab  Rotates the {\tt b} lowest (least significant) bits of
  the set {\tt s} by {\tt r} positions to the left.
  Here, {\tt s} is considered as a {\tt b}-bit number kept
  in the least significant bits of {\tt s}.
 \endtab

\noindent
{\tt\#define bitset\_RotateRightLocal(s, b, r)}

\hide\code
#define bitset_RotateRightLocal(s, b, r) do { \
   uint64_t temp = (s) >> (r); \
   (s) <<= ((b) - (r));   (s) |= temp;   (s) &= bitset_MaskLow(b); \
   } while (0)
\endcode\endhide
 \tab  Rotates the {\tt b} lowest (least significant) bits of
  the set {\tt s} by {\tt r} positions to the right.
  Here, {\tt s} is considered as a {\tt b}-bit number kept
  in the least significant bits of {\tt s}.
 \endtab


%%%%%%%%%%%%%%%%%%%%%%%%%%%%%%%%%%%%%%%%%%
\guisec{Prototypes}
\code

bitset_BitSet bitset_ReverseOrderSimple (bitset_BitSet z, int b);
\endcode
\tab Reverses the order of the {\tt b} least significant bits of {\tt z} by looping
    through the bits (simple but very inefficient). Thus, if {\tt b}${}=4$ and
    {\tt z}${} = \dots 0011$, the returned value is $\dots 1100$.
 \endtab
\code

bitset_BitSet bitset_ReverseOrder (bitset_BitSet z, int b);
\endcode
\tab Reverses the order of the {\tt b} least significant bits of {\tt z} using
    a lookup table, as implemented in \url{https://graphics.stanford.edu/~seander/bithacks.html}.
\endtab
\code

void bitset_Write (char *desc, bitset_BitSet z, int b);
\endcode
  \tab
	\hpierre{Should return the string, not print it.  Then we can do whatever
	    we want with the string.}
  Prints the string {\tt desc} (which may be empty), then writes the {\tt b}
	least significant bits of {\tt z} considered as an unsigned binary number.
  This corresponds to the {\tt b} first elements of {\tt z}.
 \endtab
\code\hide
#endif
\endhide\endcode
