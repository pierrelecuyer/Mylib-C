\documentclass[12pt]{article}

\usepackage[table]{xcolor}
%  \usepackage{crayola}
\usepackage{amsfonts}
\usepackage{amsmath}
%\usepackage{amsthm}
\usepackage[procnames]{listings}
\usepackage{url}
\usepackage{mybold}
\usepackage{mycal}
\usepackage{mymathbb}

\topmargin=-0.5in\oddsidemargin=0pt\topskip=\baselineskip
\headsep=0pt\headheight=0.4in
\textwidth=6.6 true in\textheight=9.0 true in
\parskip = 10pt
\parindent = 0pt


\lstloadlanguages{C}
\lstset{language=C,
float=tbhp,
captionpos=t,
frame=trbl,
abovecaptionskip=0.2em,
belowskip=0.3em,
basicstyle=\footnotesize\ttfamily,
stringstyle=\color{OliveGreen},
commentstyle=\color{red},
identifierstyle=\color{Bittersweet},
basewidth={0.5em},
showstringspaces=false,
framerule=0.8pt,
procnamestyle=\bfseries\color{blue},
emphstyle=\bfseries\color{Cerulean},
procnamekeys={class,extends,interface,implements}
}

\def\bs{{\tt\char'134}} 

\def\red#1{{\color{red} #1}}
\def\blue#1{{\color{blue} #1}}
\def\green#1{{\color{green} #1}}
\def\orange#1{{\color{orange} #1}}

\newcommand\guisec[1]{\vspace{10pt}%
  \noindent\hrulefill\hspace{10pt}{\bf #1}\hspace{10pt}\hrulefill
  \vspace{10pt}\nopagebreak}

\catcode`\@=11
\def\inc#1{\@partswtrue\edef\@partlist{#1}}

% Macros pour inserer des bouts de code (programmes).
% Faire  \code ...  \endcode
{\obeyspaces\gdef {\ }}
\def\setverbatim{\def\par{\leavevmode\endgraf}
            \parskip=0pt\parindent=0pt\obeylines\obeyspaces }
\chardef\other=12
\def\ttverbatim{\setverbatim\tt
       \catcode`\{=\other \catcode`\}=\other \catcode`\_=\other
       \catcode`\^=\other \catcode`\$=\other \catcode`\%=\other
       \catcode`\#=\other \catcode`\&=\other \baselineskip=11pt
       }
    % Reproduit tel quel ce qui est ecrit, en caracteres \tt.
    % On doit faire  \begingroup\ttverbatim   ....  \endgroup
\def\smallttverbatim{\ttverbatim\small\tt}
\def\code {\vfil\vfilneg\vbox\bgroup\ttverbatim}
\def\longcode {\vfil\vfilneg\bgroup\ttverbatim}
\def\smallc {\small\tt\baselineskip=9.5pt}
\def\footc {\footnotesize\tt\baselineskip=9.0pt}
\def\smallcode {\code\smallc}
\let\endcode=\egroup
\let\vcode=\code
\let\endvcode=\egroup

%  Definition d'un module ou d'une classe.
\def\ps@nomark {\def\leftmark{} \def\rightmark{}}
\def\defmodule#1 {\addcontentsline{toc}{subsection}{#1} \markboth{#1}{#1}
   \centerline {\LARGE\bf #1}\bigskip \thispagestyle{nomark}}
\def\defclass#1 {\addcontentsline{toc}{subsection}{#1} \markboth{#1}{#1}
   \centerline {\LARGE\bf #1}\bigskip \thispagestyle{nomark}}
%  Lorsqu'on veut cacher certaines choses a l'usager, faire  \hide ... \endhide
\newif\iffull\fullfalse
\def\hide{\iffull}
\let\endhide=\fi

\def\parup{\nobreak\vskip -2pt\nobreak}

\def\tab{\small\dimen9=\parindent\parindent=0pt%
   \advance\leftskip by 1.5em\parup}
\def\tabb{\small\dimen9=\parindent\parindent=0pt%
   \advance\leftskip by 3.0em\parup}
\def\tabbb{\small\dimen9=\parindent\parindent=0pt%
   \advance\leftskip by 4.5em\parup}
\def\endtab{\vskip 0.01pt\advance\leftskip by -1.5em\normalsize%
   \parindent=\dimen9}
\def\endtabb{\vskip 0.01pt\advance\leftskip by -3.0em\normalsize%
   \parindent=\dimen9}
\def\endtabbb{\vskip 0.01pt\advance\leftskip by -4.5em\normalsize%
   \parindent=\dimen9}

% Pour mettre quelque chose dans une boite double.
\def\boxit#1{\vbox{\hrule height1pt
                   \hbox{\vrule width1pt\kern3pt
                         \vbox{\kern3pt#1\kern3pt
                              }\kern3pt\vrule width1pt
                        }\hrule height1pt }}
\def\boxr#1{\hfil\vbox{\hrule height1pt
                       \hbox{\vrule width1pt\kern3pt
                              \vbox{#1}\hfil
                              \kern3pt\vrule width1pt
                             }\hrule height1pt }}

% Synonymes plus courts pour  \begin{equation}, \begin{eqnarray}, etc.
\def\eq{\equation}  \def\endeq{\endequation}
\def\eqs{\eqnarray} \def\endeqs{\endeqnarray}
\def\eqsn{\begin {eqnarray*}} \def\endeqsn{\end{eqnarray*}}

% Proof avec boite alignee a droite a la fin.
\newenvironment{myproof}{{\em Proof.}}{\hspace*{\fill}$\Box$}
% \newenvironment{proof}{{\em Proof.}}{\hspace*{\fill}$\Box$}

% Macros (avec switch) pour faire apparaitre les noms des labels dans les eqs.
% Utiliser \eqlabel au lieu de \label.  Doit laisser un espace apres le }
% Pour que les etiquettes apparaissent, faire \seeeqlabelstrue  au debut.
\newif\ifseeeqlabels\seeeqlabelsfalse
\newbox\eqlab \setbox\eqlab=\hbox {}
\def\eqlabel#1 {\global\setbox\eqlab=\hbox
   {\ifseeeqlabels {\rm (#1)} \else {} \fi } \label{#1} }
\def\@eqnnum {{\rm \box\eqlab \setbox\eqlab=\hbox {} (\theequation)}}

% Comme ci-haut pour \eqlabel, mais pour les noms des autres labels
% (Theoremes, Propositions, etc.).
% Utiliser \vislabel au lieu de \label.
% Pour que les etiquettes apparaissent, faire \seevislabelstrue  au debut.
\newif\ifseevislabels\seevislabelsfalse
\def\vislabel#1 {\ifseevislabels {\ \em (#1).\ } \else {} \fi \label{#1} }

% Pour mettre des remarques temporaires.
\newif\ifREM\REMfalse
\def\REM#1 {\ifREM  \begin{quote} \small\em #1 \end{quote} \else {\null} \fi }

% Pour avoir "running head" et no. de page en haut de page, faire  \mytwoheads
% Si on veut la date en haut de chaque page, on fait aussi  \dateheadtrue
\newif\ifdatehead\dateheadfalse
\def\mytwoheads {\pagestyle{headings}
  \topmargin=-0.4in\headheight=0.2in\headsep=0.4in
  \oddsidemargin=0.2in\evensidemargin=0in
  \def\@evenhead {{\large\bf\thepage}\quad\leftmark\hfil
    \ifdatehead\small\it\today\fi}%         Left heading
  \def\@oddhead {\ifdatehead{\small\it\kern-1em\today}\fi\hfil
    \rightmark\quad\large\bf\thepage}}%     Right heading

\catcode`\@=12


\newif\ifnotes\notestrue
%\notesfalse             %%  Uncomment this line to hide footnotes.  <----
\def\boxnote#1#2{\ifnotes\fbox{\footnote{\ }}\ \footnotetext{ From #1: #2}\fi}
\def\pierre#1{\boxnote{Pierre}{\color{red}#1}}
\def\mpierre#1{{\color{red} #1}}
\def\mp#1{{\color{red} #1}}
\def\hpierre#1{}
\def\hrichard#1{}
\def\richard#1 {\fbox {\footnote {\ }}\ \footnotetext { From Richard: #1}}

\def\hiro#1{\boxnote{Hiroshi}{\color{blue}#1}}   % Adds a footnote from Hiroshi
\newcommand{\mhiro}[1]{{\color{blue}#1}}         % Adds text in the color oh Hiroshi
\def\hhiro#1{}                                   % This is a way to comment out an addition or footnote.

\def\simon#1{\boxnote{Simon}{\color{cyan}#1}}   
\newcommand{\msimon}[1]{{\color{cyan}#1}}        
\def\hsimon#1{}                                  

\def\perhaps#1{{\color{gray} #1}}


%%%%%%%%%%%%%%%%%%%%%%%%%%%%%%%%%%%%%%%%%%%%%
\begin{document}

\begin{titlepage}

\null 
\begin {flushright} \it Last update: \today \end {flushright}

\vfill
{ \centerline {\Large\bf mylib-C Reference Manual}\bigskip\bigskip
  \centerline {\large\bf Some Basic Utilities in C }}
\vfill

\centerline {{\bf Pierre L'Ecuyer and Richard Simard}}
\medskip
\centerline {D\'epartement d'Informatique et de Recherche op\'erationnelle}
\centerline {Universit\'e de Montr\'eal}

\vfill
% \centerline {\large\bf Note}
\medskip

This document describes basic utility functions implemented in the 
C language mostly around the years 1995--2000, to be used in the software developed in the
\emph{Stochastic Simulation Laboratory} under the supervision of Pierre L'Ecuyer.
Many of these tools were originally implemented earlier in the Modula-2 language,
and used in our old software to analyze and test random number generators (RNGs) back 
in the late 1980's.  They were translated to C when some of our
software was converted from Modula-2 to C in the late 1990's.
Many of these functions may have counterparts in recent standard C libraries, 
but we keep them to avoid rewriting our code.
They are used in particular in the \texttt{TestU01} and \texttt{F2LinearGen} libraries.

This reference manual is written in LaTeX and the \texttt{.tex} source files 
are also used to produce the \texttt{.h} files of the library, to ensure that they agree with
the documentation.  Most of the function names are capitalized, but some are not,
e.g., in \texttt{bitvector} and \texttt{bitmatrix}.

In addition to the functions described here, unit-test functions are provided 
in the separate folder \texttt{mylib-unit-tests}. 
They permit one to check each function for correctness.  
\bigskip

\textbf{Contributors:}  Pierre L'Ecuyer, Richard Simard, Fran\c cois Panneton,
Francis Picard, Jean-S\'ebastien S\'en\'ecal, Simon-Olivier Laperri\`ere.

\vfill
\end{titlepage}

%%%%%%%%%%%%%%%%%%%%%%%%%%%%%%%
\pagenumbering{roman}
\tableofcontents
\pagenumbering{arabic}

\iffalse 
\defmodule {gdef}

A few macros and platform-dependent options are defined here.
These options are used by other modules to decide when 
platform-dependent functions must be commented out in the code, or not.
Most of these options are set to their true values by the program
{\it configure} in the installation process. The user may choose
to set some of them manually.
\iffalse
Each option must either be left undefined (i.e., the corresponding
macro is put to false, using ``\texttt{\#undef}) 
or can be given its proper value (using ``\texttt{\#define} commands).
An option can be defined only under certain conditions.
For example, \texttt{USE\_GMP} can be defined only if GMP
is available, \texttt{HAVE\_ERF} can be defined only if the Unix \texttt{erf}
function is available, and so on.
\fi
This module also contains functions to test near equality between numbers represented in 
\texttt{double}, and a function that prints the current host name.

%%%%%%%%%%%%%%%%%%%%%%%%%%%%%%%%%%%%%%%%%%%%%%%%%%%%
\code\hide
/* gdef.h  for ANSI C */
#ifdef __cplusplus
extern "C" {
#endif
#ifndef GDEF_H
#define GDEF_H

#include <gdefconf.h>
#include <limits.h>
#include <inttypes.h>
#include <stdint.h>

#ifdef HAVE_GMP_H
#define USE_GMP
#else
#undef USE_GMP
#endif
\endhide
\endcode

\guisec{Global macros}

\code
#define FALSE 0
#define TRUE 1

typedef int lebool;
\endcode
  \tab Defines the boolean type \texttt{lebool}, whose only possible values are
  {\tt TRUE} and {\tt FALSE}.
 \pierre{We may want to deprecate this one and use the type \texttt{bool} from \texttt{stdbool.h} instead.}
 \endtab
\code

#define DIR_SEPARATOR "/"
\endcode
  \tab Used to separate directories in the pathname of a file.
  It is \texttt{"$/$"} on {Unix-Linux} and most other platforms. 
  It may have to be set to \texttt{"$\backslash\backslash$"} on some platforms.
 \endtab
\code

#undef USE_GMP
\endcode
  \tab  Define this macro if the GNU multi-precision package GMP
  is available.  GMP is a portable library written in C for arbitrary
  precision arithmetic on integers, rational numbers, and floating-point numbers. 
	See \url{http://www.gnu.org/software/gmp/manual}. 
%	A few random number generators in library TestU01 use arbitrary large integers, and they
%  have been implemented with GMP functions. 
	If one wants to use GMP, the GMP header file (\texttt{gmp.h}) must be in the  search path 
  of the C compiler for included files, and the GMP library must be 
  linked to create executable programs.
 \endtab
\iffalse   %%%%%%
\code

#undef HAVE_MATHEMATICA
\endcode
  \tab  Define this macro if the {\em Mathematica\/}
% \footnote{
%  {\em Mathematica} is a registered trademark of {\em Wolfram Research,
%   Inc.} Web address:  \texttt{www.wolfram.com}}
    software \cite{mWOL96a}
   and the {\em MathLink} program that allows a C program to call
   functions from {\em Mathematica} are available and you want to use them.
   This is used only in module \texttt{usoft} 
   of library TestU01, where the random number generators from 
   {\em Mathematica} can be called from a C program for testing with TestU01.

   When a C program uses {\em Mathematica}, it must be compiled with the
    options
   \texttt{-I\$MATHINC -L\$MATHLIB -lML}, where 
   \texttt{\$MATHINC} is the path to the header file \texttt{mathlink.h} and 
   \texttt{\$MATHLIB} is the path to the {\em MathLink\/} library \texttt{libML.a}.
   For example, in the environment of our lab, both 
   \texttt{\$MATHINC} and \texttt{\$MATHLIB} must be set to

   \url{<dir>/mathematica/5.0/linux/AddOns/MathLink/DeveloperKit/Linux/CompilerAdditions}.

  To run a main program named \texttt{tulip} on a Unix/Linux platform
  that calls {\em  Mathematica} functions, one may use
 \begin{verse} 
  \texttt{tulip -linkname 'math -mathlink' -linklaunch}.
 \end{verse}
  \endtab
\fi  %%%%%%

%  \newpage

%%%%%%%%%%%%%%%%%%%%%%%%%%%%
\guisec{Host machine}
\code

void gdef_GetHostName (char machine[], int n);
\endcode
  \tab Returns in \texttt{machine} the host name. 
  Will copy at most $n$ characters, so the array \texttt{machine[]}
  should have a size $\ge n$. This is useful, for example,
  to get the name of the machine on which a program is running.
 \hpierre{Does this work on any type of system or only on Linux?}
 \hrichard{Ceci fonctionne avec presque tous les syst\`emes Unix ou Linux et ceux qui
  respectent le standard POSIX.}
  \endtab
\code

void gdef_WriteHostName (void);
\endcode
  \tab Prints the name of the machine on which a program is running.
   This should work on any Unix or Linux machine.
  \endtab
\hide
\code
#endif
#ifdef __cplusplus
}
#endif
\endcode
\endhide

%%%%%%%%%%%%%%%%%%%%%%%%%%%%%%%%%%%%%%%%%%%%%%%%%%

\defmodule {util}

This module offers macros and functions used for testing, print error messages,
allocate dynamic memory, and read/write Boolean variables.
Some of these ``functions'' are implemented as macros, for better speed.
The module also contains safe prototype functions to open and close files.


%%%%%%%%%%%%%%%%%%%%%%%%%%%%%%%%%%%%%%%%%%%%%%%%%%%%
\bigskip\hrule

\code\hide
/* util.h  for ANSI C */
#ifndef UTIL_H
#define UTIL_H
\endhide
#include "gdef.h"
#include <stdio.h>
#include <stdlib.h>
\endcode

%%%%%%%%%%%%%%%%%%%%%%%%%%%%%%%%%%%%%%%%%%
\guisec{Macros}

\noindent
{\tt util\_Error (s)};

 \tab  Print the string {\tt s}, then stop the program.
 \endtab
\code
\hide
#define util_Error(s) do { \
   printf ("\n\n******************************************\n"); \
   printf ("ERROR in file %s   on line  %d\n\n", __FILE__, __LINE__); \
   printf ("%s\n******************************************\n\n", s); \
   exit (EXIT_FAILURE); \
   } while (0)
\endhide
\endcode

\noindent
{\tt util\_Assert (Assertion, s)};

 \tab  If {\tt lebool Assertion} is {\tt FALSE} (= 0),
  this macro prints the string {\tt s} and stops the program.
 \endtab
\code
\hide
#define util_Assert(Assertion,s) if (!(Assertion)) util_Error(s)
\endhide
\endcode

\noindent
{\tt util\_Warning (Condition, s)};

 \tab  If {\tt lebool Condition} is {\tt TRUE} ($\not = 0$),
 this macro prints the string {\tt s}.
 \endtab
\code
\hide
#define util_Warning(Cond,s) do { \
   if (Cond) { \
      printf ("*********  WARNING "); \
      printf ("in file  %s  on line  %d\n", __FILE__, __LINE__); \
      printf ("*********  %s\n", s); } \
   } while (0)
\endhide
\endcode

\noindent
{\tt util\_Max (x, y)};

 \tab  Returns the largest of the two numbers {\tt  x}, {\tt y}.
 \endtab
\code
\hide
#define util_Max(x,y) (((x) > (y)) ? (x) : (y))
\endhide
\endcode

\noindent
{\tt util\_Min (x, y)};

 \tab  Returns the smallest of the two numbers {\tt  x}, {\tt y}.
 \endtab
\code
\hide
#define util_Min(x,y) (((x) < (y)) ? (x) : (y))
\endhide
\endcode


%%%%%%%%%%%%%%%%%%%%%%%%%%%%%%%%%%%%%%%%%%
\guisec{Prototypes}

\code

lebool util_nearEqual (double x, double y, double eps);
\endcode
 \tab  Tests if the two floating-point numbers $x$ and $y$ are nearly equal.
 Returns {\tt TRUE} if $|x-y| < \texttt{eps}$,
 and to {\tt FALSE} otherwise.
 \endtab
\code

lebool util_nearEqualRel (double x, double y, double eps);
\endcode
 \tab  Tests if the two floating-point numbers $x$ and $y$ are nearly equal,
  relatively to the value of $|y|$.
  Returns {\tt TRUE} if $|x-y| < \texttt{eps} |y|$, and to {\tt FALSE} otherwise.
  This is often used if the result $x$ of a calculation is supposed to equal
	a known value $y$, and we want to test if the calculation appears correct.
	One might then take \texttt{eps} somewhere around $10^{-12}$ to $10^{-15}$, for example.
 \hpierre{See also \url{https://bitbashing.io/comparing-floats.html} for more
  sophisticated comparison methods of floating-point numbers.}
 \endtab
\code

lebool util_nearEqualDefault (double x, double y);
\endcode
 \tab  Tests if the two floating-point numbers $x$ and $y$ are nearly equal,
  relatively to the value of $|y|$ by calling {\tt util\_nearEqualRel} and by
  using a default value of 1.0E-12 for the bound \texttt{eps} on the relative
  error.
 \endtab
\code

FILE * util_Fopen (const char *name, const char *mode);
\endcode
  \tab
  Calls {\tt fopen} (from {\tt stdio.h}) with the same arguments, but checks for errors.
  Opens or creates file with name {\tt name} in mode {\tt mode}. Returns a pointer to
  FILE that is associated with the stream. If {\tt name} cannot be accessed, the program
  stops.
 \endtab
\code

int util_Fclose (FILE *stream);
\endcode
  \tab
   Calls {\tt fclose} (from {\tt stdio.h}) with the same arguments, but checks for errors.
   Closes the file associated with {\tt stream}. If the file is successfully
   closed, {\tt 0}
   is returned. If an error occurs or the file was already closed, {\tt EOF} is returned.
 \endtab
\code

int util_GetLine (FILE *file, char *Line, char c);
\endcode
  \tab
  Reads a line of data from {\tt file}. Blank lines and comments are
  ignored. A comment is any line whose first non-whitespace character
  is {\tt c}. If the character {\tt c} appears anywhere on a line that is
  not a comment, then  {\tt c} and the rest of the line are ignored too.
  The function returns $-1$ if end-of-file or an error is encountered,
  otherwise it returns 0.
  \endtab
\code

void util_ReadLn (FILE *f);
\endcode
  \tab
  Skips the remainder of the current line of FILE \texttt{f} and moves the current position
  of the pointer to FILE \texttt{f} to the beginning of the next line.
  \endtab
\code

void util_ReadBool (char S[], lebool *x);
\endcode
  \tab
  Reads a {\tt lebool} value from string {\tt S} and returns it in  $x$.
  The possible values are  {\tt TRUE} and  {\tt FALSE}.
  \endtab
\code

void util_WriteBool (lebool x, int d);
\endcode
  \tab
  Writes the value of $x$ in a field of width $d$. If $d < 0$,
  $x$ is left-justified, otherwise right-justified.
  \endtab
\code

void * util_Malloc (size_t size);
\endcode
  \tab
  Calls {\tt malloc} (from {\tt stdlib.h}) with same arguments, but checks for errors.
  Allocates memory large enough to hold an object of size {\tt size}.
	A successful call returns the base address of the allocated space, otherwise the
  programs stops. The standard type {\tt size\_t} is defined in {\tt stdio.h}.
 \endtab
\code

void * util_Calloc (size_t dim, size_t size);
\endcode
  \tab
  Calls {\tt calloc} (from {\tt stdlib.h}) with same arguments, but checks for errors.
  Allocates memory large enough to hold an array of {\tt dim}
  objects each of size {\tt size}. A successful call
  returns the base address of the allocated space, otherwise the programs
  stops. The standard type {\tt size\_t} is defined in {\tt stdio.h}.
 \endtab
\code

void * util_Realloc (void *ptr, size_t size);
\endcode
  \tab Calls {\tt realloc} (from {\tt stdlib.h}) with same arguments, but
  checks for errors.
  Takes a pointer to a memory region previously allocated and referenced by
  {\tt ptr}, then changes its
  size to {\tt size} while preserving its content.
% The function attempts to  keep the same base address for the block,
% but if it is not possible, it allocates a new block of  memory,
% copying the relevant portion of the old block and deallocating it.
  A successful call
  returns the base address of the resized (or new) space, otherwise the
  programs stops. The standard type {\tt size\_t} is defined in {\tt stdio.h}.
 \endtab
\code

void * util_Free (void *p);
\endcode
  \tab Calls {\tt free (p)} (from {\tt stdlib.h}) to free memory
	allocated by {\tt util\_Malloc},
   {\tt util\_Calloc} or {\tt util\_Realloc}. Always returns a {\tt NULL} pointer.
 \endtab
\code\hide
#endif
\endhide\endcode

\defmodule {num}

This module provides some useful constants and basic tools to
manipulate numbers represented in different forms.

\bigskip\hrule

\code\hide
/* num.h for ANSI C */

#ifndef NUM_H
#define NUM_H
\endhide
#include "gdef.h"
\endcode

%%%%%%%%%%%%%%%%%%%%%%%%%%%%%%%%%%%%%%%%%%
\guisec{Constants}
\code

#define num_Pi     3.14159265358979323846
\endcode
  \tab The number $\pi$.
  \endtab
\code

#define num_ebase  2.7182818284590452354
\endcode
  \tab The Euler number $e$.
  \endtab
\code

#define num_Rac2   1.41421356237309504880
\endcode
  \tab $\sqrt{2}$, the square root of 2.
  \endtab
\code

#define num_1Rac2  0.70710678118654752440
\endcode
  \tab $1/\sqrt{2}$.
  \endtab
\code

#define num_Ln2    0.69314718055994530941
\endcode
  \tab $\ln(2)$, the natural logarithm of 2.
  \endtab
\code

#define num_1Ln2   1.44269504088896340737
\endcode
  \tab $1 / \ln(2)$.
  \endtab
\code

#define num_MaxIntDouble   9007199254740992.0
\endcode
  \tab Largest integer $n_0 = 2^{53}$ such that all integers
  $n \le n_0$ are represented  exactly as a {\tt double}.
  \endtab


%%%%%%%%%%%%%%%%%%%%%%%%%%%%%%%%%%%%%%%%%%
\guisec{Precomputed powers}
\code

#define num_MaxTwoExp   64
\endcode
  \tab Powers of 2 up to {\tt num\_MaxTwoExp} are stored exactly
  in the array {\tt num\_TwoExp}.
  \endtab
\code

extern double num_TwoExp[];
\endcode
  \tab  Contains precomputed powers of 2.
  One has {\tt num\_TwoExp[i]} $= 2^i$ for $0 \le i \le$
  {\tt num\_MaxTwoExp}.
\endtab
\code

#define num_MAXTENNEGPOW   16
\endcode
  \tab Negative powers of 10 up to {\tt num\_MAXTENNEGPOW} are stored
  in the array {\tt num\_TENNEGPOW}.
  \endtab
\code

extern double num_TENNEGPOW[];
\endcode
 \tab Contains the precomputed negative powers of 10.
   One has {\tt TENNEGPOW[j]}$ = 10^{-j}$, for $j=0,\ldots,$
 {\tt num\_MAXTENNEGPOW}.
\endtab


%%%%%%%%%%%%%%%%%%%%%%%%%%%%%%%%%%%%%%%%%%
\guisec{Prototypes}
\code

#define num_Log2(x) (num_1Ln2 * log(x))
\endcode
  \tab Gives the logarithm of $x$ in base 2.
  \endtab
\code

int64_t num_Round64 (double x);
\endcode
  \tab Rounds $x$ to the nearest ({\tt int64\_t}) integer and returns it.
  \endtab
\code

double num_RoundD (double x);
\endcode
  \tab Rounds $x$ to the nearest ({\tt double}) integer and returns it.
  \endtab
\code

int num_IsNumber (char S[]);
\endcode
\tab  Returns {\tt 1} if the string {\tt S} begins with a number
   (with the possibility of spaces and a $+/-$ sign
   before the number). For example, `` + 2'' and ``4hello''
   return {\tt 1}, while ``$-+2$'' and ``hello'' return  {\tt 0}.
\endtab
\code

void num_IntToStrBase (int64_t k, int b, char S[]);
\endcode
  \tab Returns in {\tt S} the string representation of {\tt k} in base {\tt b}.
  \endtab
\code

void num_Uint2Uchar (unsigned char output[], uint64_t input[], int L);
\endcode
  \tab Transforms the $L$ 64-bit unsigned integers contained in {\tt input} into
  $8L$ characters and puts them into {\tt output}. The order is such that
  the 8 most significant bits of {\tt input[0]} will be in {\tt output[0]},
  the 8 least significant bits of {\tt input[0]} will be in {\tt output[7]},
  and the 8 least significant bits of {\tt input[L-1]} will be in
  {\tt output[8L-1]}. Array {\tt output} must have at least $8L$ elements.
  \endtab
\code

void num_WriteD (double x, int i, int j, int k);
\endcode
  \tab  Writes {\tt x} to current output.  Uses a total of at least {\tt i}
   positions (including the sign and point when they appear),
   {\tt j} digits after the decimal point and at least {\tt k}
   significant digits.   The number is rounded if necessary.
   If there is not enough space to print the number in decimal notation
   with at least {\tt k} significant digits
   ({\tt j} or {\tt i} is too small), it will be printed in scientific
   notation with at least {\tt k} significant digits.
   In that case, {\tt i} is increased if necessary.
   Restriction: {\tt j} and {\tt k} must be strictly smaller than {\tt i}.
 \endtab
\code

void num_WriteBits (uint64_t x, int k);
\endcode
 \tab Writes {\tt x} in base 2 in a field of at least $\max\{64, |k|\}$ positions.
  If $k>0$, the number will be right-justified, otherwise left-justified.
 \endtab
\code

int64_t num_MultMod (int64_t a, int64_t s, int64_t c, int64_t m);
\endcode
\tab  Returns $(as + c) \mod m$.  Uses the decomposition technique
  of \cite{rLEC91a} to avoid overflow.
	Assumptions: $\max(a,s,c) < m < 2^{63}$.
%  This function can handle higher values than {\tt num\_MultModLL}.
\endtab
\code

int64_t num_MultModDirect (int64_t a, int64_t s, int64_t c, int64_t m);
\endcode
\tab  Returns $(as + c) \mod m$.  Uses direct multiplication and addition.
  Assumptions: $m < 2^{63}$ and $as+c < 2^{63}$.
\endtab
\code

double num_MultModDouble (double a, double s, double c, double m);
\endcode
 \tab  Returns $(as+c) \mod m$, assuming that
  $a$, $s$, $c$, and $m$ are all {\em integers\/} less than $2^{35}$
  (represented exactly).
  Works under the assumption that all positive integers less than
  $2^{53}$ are represented exactly in floating-point (in {\tt double}).
\endtab
\code

int64_t num_InvEuclid (int64_t m, int64_t z);
\endcode
 \tab  This function computes the inverse $z^{-1}\bmod m$ by the
  modified Euclid algorithm (see \cite[p. 325]{iKNU81a}) and returns
  the result. If the inverse does not exist, returns 0. This function
  can handle higher values than {\tt num\_InvEuclid32}.
\endtab
\code

long num_InvEuclid32 (long m, long z);
\endcode
 \tab  Same as \texttt{num\_InvEuclid}, but for 32-bit integers.
\endtab
\code

uint64_t num_InvExpon (int e, uint64_t z);
\endcode
 \tab
  This function computes the inverse  $z^{-1} \bmod 2^e$
  by exponentiation  and returns the result. If the inverse does not  exist, returns 0.
  Restriction: $e \le 64$.
%	not larger than the number of bits in an {\tt unsigned long}.
 \endtab
\code

long num_InvExpon32 (int e, long z);
 \endcode
 \tab  Same as \texttt{num\_InvExpon}, but for 32-bit integers.
  Restriction: $e \le 32$.
\endtab
\code

uint64_t num_gcd (uint64_t a, uint64_t b);
 \endcode
 \tab  Returns the greatest common divisor gcd(\texttt{a,b}).
\endtab
\code

long num_gcd32 (long a, long b);
 \endcode
 \tab  Same as \texttt{num\_gcd}, but for 32-bit integers.
\endtab
\code

lebool num_isMersennePrime (int e);
 \endcode
 \tab  Returns \texttt{TRUE} iff $2^e-1$ is a Mersenne prime for $e \le 110503$.
\endtab
\code\hide
#endif
\endhide\endcode

\defmodule {num2}

This module provides procedures to compute certain numerical
quantities such as factorials, combinations, Stirling numbers,
Bessel functions, gamma functions, and so on.
These functions are more involved than those provided by {\tt num}.

\bigskip\hrule

\code\hide
/* num2.h for ANSI C */

#ifndef NUM2_H
#define NUM2_H
\endhide
#include "gdef.h"
#include <math.h>
\endcode

%%%%%%%%%%%%%%%%%%%%
\guisec{Prototypes}
\code

double num2_Factorial (int n);
\endcode
 \tab The factorial function. Returns the value of $n!$
\endtab
\code

double num2_LnFactorial (int n);
\endcode
 \tab Returns the value of $\ln (n!)$, the natural logarithm of the
 factorial of  $n$. Gives at least 16 decimal digits of precision
  (relative error $< 0.5\times 10^{-15}$)
\endtab
\code

double num2_Combination (int n, int s);
\endcode
  \tab Returns the value of $\binom{n}{s}$, the number of different combinations
   of $s$ objects amongst $n$. %  Uses an algorithm that prevents overflows
  % (when computing factorials), if possible.
 \endtab
\code

#ifdef HAVE_LGAMMA
#define num2_LnGamma lgamma
#else
   double num2_LnGamma (double x);
#endif
\endcode
  \tab Calculates the natural logarithm of the gamma function  $\Gamma(x)$
   at {\tt x}. Our {\tt num2\_LnGamma} gives 16 decimal digits
   of precision, but is implemented only for $x>0$.
   The function {\tt lgamma} is from the standard math library \texttt{math.h}.
  \endtab
\code

double num2_Digamma (double x);
\endcode
\tab Returns the value of the logarithmic derivative of the Gamma function
   $\psi(x) = \Gamma'(x) / \Gamma(x)$.
\endtab
\code

#ifdef HAVE_LOG1P
#define num2_log1p log1p
#else
   double num2_log1p (double x);
#endif
\endcode
  \tab Returns an approximation of {\tt log}$(1 + x)$ which is accurate for small $x$. 
	The function {\tt log1p} is from the standard math library \texttt{math.h}.
  \endtab
\code

void num2_CalcMatStirling (double *** M, int m, int n);
\endcode
 \tab Calculates the Stirling numbers of the second kind and returns them in matrix $M$ with
 \eq
   M[i,j] = \left\{\begin{array}{c}j \\ i\end{array}\right\}
     \quad \mbox { for $0\le i\le m$ and $0\le i\le j\le n$}.
                                                        \label{Stirling2}
 \endeq
  See \cite{iKNU73a}, Section 1.2.6.
  Our matrix \texttt{M} is the transpose of that in \cite{iKNU73a}.
  This procedure allocates memory for the two-dimensional matrix $M$,
  and fills it with the values of Stirling numbers;
  the memory should be released later by invoking {\tt num2\_FreeMatStirling}.
 \endtab
\code

void num2_FreeMatStirling (double *** M, int m);
\endcode
  \tab Frees the memory space used by the Stirling matrix created by calling
  {\tt num2\_CalcMatStirling}. The parameter {\tt m}
  must be the same as the {\tt m} in  {\tt num2\_CalcMatStirling}.
  \endtab
\code

double num2_VolumeSphere (double p, int t);
\endcode
\tab Calculates the volume $V$ of a sphere of radius 1 in $t$ dimensions
  using the norm $L_p$, according to the formula
$$
       V = \frac{\left[2 \Gamma(1 + 1/p)\right]^t}
             {\Gamma\left(1 + t/p\right)}, \qquad p > 0,
$$
  where $\Gamma$ is the well-known gamma function.
  The case of the sup norm $L_\infty$ is
  obtained by choosing $p=0$.
  Restrictions: $p\ge 0$ and $t\ge 1$.
  \endtab
\code

double num2_EvalCheby (const double a[], int n, double x);
\endcode
\tab Evaluates a series of Chebyshev polynomials $T_j$, at point
  $x \in [-1, \;1]$, using the method of Clenshaw \cite{mCLE62a},
   i.e. calculates and  returns
  $$
    y = \frac{a_0}2 + \sum_{j=1}^n a_j T_j(x),
  $$
where $a_0,\dots,a_n$	are given in array \texttt{a[0..n]}.
\endtab
\code

double num2_BesselK025 (double x);
\endcode
\tab Returns the value of $K_{1/4}(x)$, where $K_{\nu}$ is the modified
  Bessel's function of the second kind.
  The relative error on the returned value is less than
  $0.5\times 10^{-6}$ for $x > 10^{-300}$.
\endtab
\code\hide
#endif
\endhide\endcode






\defmodule {mystr}

This module offers some tools for the manipulation of character strings. 

\bigskip\hrule
\code\iffalse
/* mystr.h for ANSI C */

#ifndef MYSTR_H
#define MYSTR_H
\fi

void mystr_Delete (char S[], unsigned int index, unsigned int len);
\endcode
 \tab  Deletes {\tt len} characters from S, starting at position
 {\tt index}.
 \endtab
\code


void mystr_Insert (char Res[], char Source[], unsigned int Pos);
\endcode
 \tab  Inserts the string {\tt Source} into {\tt Res}, 
  starting at position {\tt Pos}.
 \endtab
\code


void mystr_ItemS (char R[], char S[], const char T[], unsigned int N);
\endcode
 \tab  Returns in R the N-th substring of S (counting from 0).
  Substrings are delimited by any character from the set T.
 \endtab
\code


int mystr_Match (char Source[], char Pattern[]);
\endcode
 \tab  Returns {\tt 1} if the string {\tt Source} matches the 
  string {\tt Pattern}, and {\tt 0} otherwise.
  The characters ``?'' and ``*'' are recognized as wild characters in the
  string {\tt Pattern}.
 \endtab
\code


void mystr_Slice (char R[], char S[], unsigned int P, unsigned int L);
\endcode
 \tab  Returns in {\tt R} the substring in {\tt S} beginning at
  position {\tt P} and of length {\tt L}.
 \endtab
\code


void mystr_Subst (char Source[], char OldPattern[], char NewPattern[]);
\endcode
 \tab  Searches for the string {\tt OldPattern} in the string {\tt Source}, 
 and replaces its first occurence with {\tt NewPattern}.
 \endtab
\code


void mystr_Position (char Substring[], char Source[], unsigned int at,
                     unsigned int * pos, int * found);
\endcode
 \tab  Searches for the string {\tt Substring} in the string {\tt Source},
 starting at position {\tt at}, and returns the position of its first
 occurence in {\tt pos}.
 \endtab
\code\hide
#endif
\endhide\endcode

\defmodule {addstr}

The functions described here are convenient tools for constructing
character strings that contain a series of numeric parameters, with their values.
For example, suppose one wishes to put 
``{\tt LCG with m = 101, a = 12, s = 1}'' in the string {\tt str}, 
where the actual 
numbers 101, 12, and 1 must be taken as the values of {\tt uint64\_t}
variables {\tt m}, {\tt a}, and {\tt s}.  
This can be achieved by the instructions:
\vcode

   strcpy (str, "LCG with ");
   addstr\_uint64 (str, " m = ", m);
   addstr\_uint64 (str, ", a = ", m);
   addstr\_uint64 (str, ", s = ", s);
\endvcode

Each function {\tt addstr\_... (char *to, const char *add, ...)}
first appends the string {\tt add} to the string {\tt to}, then
appends to it a character string representation of the number 
(or array of numbers) specified by its last parameter.
In the case of an array of numbers (e.g., {\tt addstr\_array\_uint64}),
the parameter {\tt high} specifies the size of the array, and the
elements {\tt [0..high-1]} are added to {\tt str}.
The {\tt ...\_int64} and {\tt ...\_uint64} versions are for 64-bit integers.
In all cases, the string {\tt to} should be large enough to accomodate
what is appended.


%%%%%%%%%%%%%
\bigskip\hrule
\code\hide
/*  addstr.h  for ANSI C  */
#ifndef ADDSTR_H
#define ADDSTR_H
\endhide
#include "gdef.h"
\endcode

%%%%%%%%%%%%%%%%%%%%%%%%%%%%%%%%%%%%%%%%%%
\guisec{Prototypes}
\code

void  addstr_int (char *to, const char *add, int n);

void  addstr_uint (char *to, const char *add, unsigned int n);

void  addstr_long (char *to, const char *add, long n);

void  addstr_ulong (char *to, const char *add, unsigned long n);

void  addstr_int64 (char *to, const char *add, int64_t n);

void  addstr_uint64 (char *to, const char *add, uint64_t n);

void  addstr_double (char *to, const char *add, double x);

void  addstr_char (char *to, const char *add, char c);

void  addstr_bool (char *to, const char *add, int b);
\endcode

\code

void  addstr_array_int (char *to, const char *add, int high, int []);

void  addstr_array_uint (char *to, const char *add, int high,
                        unsigned int []);

void  addstr_array_long (char *to, const char *add, int high, long []);

void  addstr_array_ulong (char *to, const char *add, int high,
                         unsigned long []);

void  addstr_array_int64 (char *to, const char *add, int high, int64_t []);

void  addstr_array_uint64 (char *to, const char *add, int high, uint64_t[]);

void  addstr_array_double (char *to, const char *add, int high, double []);

void  addstr_array_char (char *to, const char *add, int high, char []);
\hide
#endif
\endhide\endcode


\defmodule {tables}

This module provides an implementation of variable-sized arrays (matrices),
and procedures to manipulate them.
The advantage is that the size of the array needs not be known
at compile time; it can be specified only during the program execution.
There are also procedures to sort arrays,  to
print  arrays in different formats,
and a few tools for hashing tables.
The functions {\tt tables\_CreateMatrix...} and
{\tt tables\_DeleteMatrix...} manage memory allocation for
these dynamic matrices.

As an illustration, the following piece of code declares and creates
a $100\times 500$ table of floating point numbers, assigns a value
to one table entry, and eventually deletes the table:
  \begin{verse}{\tt
    double ** T;\\
    T = tables\_CreateMatrix\_double (100, 500);\\
    T[3][7] = 1.234;\\
    \dots \\
    tables\_DeleteMatrix\_double (\&T);
  }\end{verse}

%%%%%%%%%%%%%
\bigskip\hrule

\code\hide
/* tables.h for ANSI C */
#ifndef TABLES_H
#define TABLES_H
\endhide
#include "gdef.h"
\endcode

%%%%%%%%%%%%%%%%%%%%%%%%%%%%%%%%%%%%%%%%%%
\guisec{Printing styles}
\code

typedef enum {
   tables_Plain,
   tables_Mathematica,
   tables_Matlab
   } tables_StyleType;
\endcode
  \tab Printing styles for matrices.
  \endtab

%%%%%%%%%%%%%%%%%%%%%%%%%%%%%%%%%%%%%%%%%%
\guisec{Functions to create, delete, sort, and print tables}
\code

long ** tables_CreateMatrix_long  (int m, int n);
unsigned long ** tables_CreateMatrix_ulong (int m, int n);
int64_t ** tables_CreateMatrix_int64  (int m, int n);
uint64_t long ** tables_CreateMatrix_uint64 (int m, int n);
double ** tables_CreateMatrix_double  (int m, int n);
\endcode
  \tab Allocates contiguous memory for a dynamic 
  matrix of {\tt m} rows and {\tt n} columns. Returns the base
  address of the allocated space.
  \endtab
\code

void tables_DeleteMatrix_long  (long *** mat);
void tables_DeleteMatrix_ulong (unsigned long *** mat);
void tables_DeleteMatrix_int64  (int64_t *** mat);
void tables_DeleteMatrix_uint64 (uint64_t *** mat);
void tables_DeleteMatrix_double  (double *** mat);
\endcode
  \tab Releases the memory used by the matrix {\tt mat}
  (see {\tt tables\_CreateMatrix}) passed by reference; i.e., using the {\tt \&} symbol. 
  Then, {\tt mat} is set to {\tt NULL}.
  \endtab
\code

void tables_CopyTab_long (long mat1[], long mat2[], int n1, int n2);
void tables_CopyTab_int64 (long mat1[], long mat2[], int n1, int n2);
void tables_CopyTab_uint64 (long mat1[], long mat2[], int n1, int n2);
void tables_CopyTab_double (double mat1[], double mat2[], int n1, int n2);
\endcode
  \tab Copies {\tt mat1[n1..n2]} in {\tt mat2[n1..n2]}.
	\pierre{Oups... I think in C, the copy order is usually the other way!  Check this. }
  \endtab
\code

void tables_QuickSort_long (long mat[], int n1, int n2);
void tables_QuickSort_int64 (int64_t mat[], int n1, int n2);
void tables_QuickSort_uint64 (uint64_t mat[], int n1, int n2);
void tables_QuickSort_double (double mat[], int n1, int n2);
\endcode
 \tab Sort the tables {\tt mat[n1..n2]} in increasing order.
 \endtab
\code

void tables_WriteTab_long (long mat[], int n1, int n2, int k, int p, 
                           char desc[]);
void tables_WriteTab_int64 (int64_t mat[], int n1, int n2, int k, int p, 
                            char desc[]);
void tables_WriteTab_uint64 (uint64_t mat[], int n1, int n2, int k, int p, 
                             char desc[]);
\endcode
 \tab  Write the elements {\tt n1} to {\tt n2} of table {\tt mat},
  {\tt k} per line, {\tt p} positions per element.
  If  {\tt k} = 1, the index will also be printed. {\tt desc}
  contains a description of the table.
 \endtab
\code

void tables_WriteTab_double (double mat[], int n1, int n2, int k, 
                             int p1, int p2, int p3, char desc[]);
\endcode
 \tab  Writes the elements {\tt n1} to {\tt n2} of table {\tt mat},
  {\tt k} per line, with at least {\tt p1} positions per element,
  {\tt p2} digits after the decimal point, and at least  {\tt p3} significant digits.
   If {\tt k} = 1, the index
  will also be printed. {\tt desc} contains a description of the table.
 \endtab\code

void tables_WriteSubmatrix_double (double** mat, int i1, int i2, int j1, int j2,
                             int w, int p, tables_StyleType style, char name[]);
\endcode
 \tab Writes the submatrix with lines 
   {\tt i1} $\le i \le $ {\tt i2} and columns 
   {\tt j1} $\le j \le $ {\tt j2} of the matrix {\tt mat} with format
   {\tt style}. The elements are printed in {\tt w}
   positions with a precision of {\tt p} digits. {\tt name} is
   an identifier for the submatrix.
  
   For {\tt Matlab}, the file containing the printed submatrix should have 
	 the extension {\tt .m}.
   For example, if it is named {\tt poil.m}, it will be accessed by the
   simple call {\tt poil} in {\tt Matlab}.
   For {\tt Mathematica}, if the file is named {\tt poil},
   it will be read using {\tt << poil;}.
 \endtab
\code

void tables_WriteSubmatrix_long (long** mat, int i1, int i2, int j1, int j2, 
                                 int w, tables_StyleType style, char name[]);
void tables_WriteSubmatrix_int64 (int64** mat, int i1, int i2, int j1, int j2,
                                  int w, tables_StyleType style, char name[]);
void tables_WriteSubmatrix_uint64 (uint64** mat, int i1, int i2, int j1, int j2, 
                                   int w, tables_StyleType style, char name[]);
\endcode
 \tab Similar to {\tt tables\_WriteMatrix\_double}.
 \endtab
\code

int64_t tables_HashPrime (int64_t n, double load);
\endcode
  \tab Returns a prime number $p$ to be used as the size 
   (the number of elements) of a hashing table.
   $p$ will be such that the load factor $n/p$ do not exceed {\tt load}.
   If {\tt load} is small, an important part of the table will be unused; that
   will accelerate searches and insertions.
   This function uses a small sequence of prime numbers; the real load factor
   may be significantly smaller than {\tt load} because only a limited
   number of prime numbers are in the table. In case of failure, returns $-1$.
 \endtab
\code

long tables_HashPrime32 (long n, double load);
\endcode
  \tab Same as \texttt{tables\_HashPrime}, but with 32-bit integers.
 \endtab
\code\hide
#endif
\endhide
\endcode

\defmodule {bitset}

This module defines sets of bits and useful operations for such sets.
Some of these operations are implemented as macros.
Each bit set is stored as a 64-bit unsigned integer, whose bits are simply interpreted
as indicators of which elements belong to the set.
The bits are numbered from 0 to 63, and 
\emph{bit 0 is the least significant bit} of this word,
which means that the bits are numbered from right to left.
\hpierre{Here the bits are numbered from the right, whereas they are 
  numbered from the left in \texttt{bitvectors}.}%
%	Are there bitsets like this in other libraries?}
If bit $j$ is 1, then element $j$ is a member of the set, otherwise it is not.
Other operations not described here can also be applied directly
to the \texttt{uint64\_t} variable. 


\code\hide
/* bitset.h  for ANSI C */
#ifndef BITSET_H
#define BITSET_H
#include "gdef.h"
\endhide\endcode


%%%%%%%%%%%%%%%%%%%%%%%%%%%%%%%%%%%%%%%%%%
\guisec{Constants}
\code

const uint64_t bitset_ONE = 1;
const uint64_t bitset_ALLONES = 18446744073709551615;   
\endcode
\tab
  The two constants 1 and $2^{64}-1$ (a bitset with all ones).
\endtab
\iffalse  %%%%
\code

extern uint64_t bitset_MASK1[];
\endcode
 \tab {\tt bitset\_MASK1[j]} has only \emph{bit} $j$ set to 1 and all other 
  bits set to 0. 
  \pierre{Not sure if it is worthwhile to store and retrieve all these constants. 
	  See \url{http://c-faq.com/misc/bitsets.html} and 
    \url{https://github.com/iplinux/x11proto-trap/blob/master/xtrapbits.h}
		for alternatives.}
 \endtab
\code

extern uint64_t bitset_MASK[];
\endcode
 \tab {\tt bitset\_MASK[j]} has all \emph{the first $j$ bits} set to 1 and all other 
  bits set to 0.  
 \endtab
\fi  %%%%%

%%%%%%%%%%%%%%%%%%%%%%%%%%%%%%%%%%%%%%%%%%
\guisec{Types}
\code

typedef uint64_t bitset_BitSet;
\endcode
 \tab  A set of bits. The bits are numbered starting from 0 for the least significant bit. 
 \endtab


%%%%%%%%%%%%%%%%%%%%%%%%%%%%%%%%%%%%%%%%%%
\guisec{Macros}

\code

#define bitset_Mask1(b) (bitset_ONE << b)
\endcode
 \tab Gives a bit set with only bit $b$ set to 1 and all other bits set to 0. 
 \endtab
\code

#define bitset_MaskLow(b) (bitset_ALLONES >> (64-b))
\endcode
 \tab Gives a bit set with all \emph{the lowest (rightmost) $b$ bits} set to 1 
  and all other bits set to 0.  
 \endtab
\code

#define bitset_SetBit(s, b) ((s) |= (bitset_Mask1(b)));
\endcode
 \tab  Sets bit {\tt b} in set  {\tt s}  to 1.
 \endtab
\code

#define bitset_ClearBit(s, b) ((s) &= ~(bitset_Mask1(b)))
\endcode
 \tab  Sets bit {\tt b} in set  {\tt s}  to 0.
 \endtab
\code

#define bitset_FlipBit(s, b) ((s) ^= (bitset_Mask1(b)))
\endcode
 \tab  Flips bit {\tt b} in set {\tt s};  thus,
        $0 \rightarrow 1$ and $1 \rightarrow 0$.
 \endtab
\code

#define bitset_GetBit(s, b)  ((s) & (bitset_Mask1(b)) ? 1 : 0)
\endcode
 \tab  Returns the value of bit {\tt b} in set {\tt s} (0 or 1).
 \endtab
\code

#define bitset_RotateLeft(s, r)  ((s << r) | (s >> (64 - r)))
\endcode
 \tab  Rotates the bit set {\tt s} by {\tt r} positions to the left.
 \endtab
\code

#define bitset_RotateRight(s, r)  ((s >> r) | (s << (64 - r)))
\endcode
 \tab  Rotates the bit set {\tt s} by {\tt r} positions to the right.
\endtab

\noindent
{\tt\#define bitset\_RotateLeftLocal(s, b, r)}  ...

\hide\code
#define bitset_RotateLeftLocal(s, b, r) do { \
   uint64_t temp = (s) >> ((b) - (r)); \
   (s) <<= (r);   (s) |= temp;   (s) &= bitset_MaskLow(b); \
   } while (0)
\endcode\endhide
\tab  Rotates the {\tt b} lowest (least significant) bits of 
  the set {\tt s} by {\tt r} positions to the left.
  Here, {\tt s} is considered as a {\tt b}-bit number kept
  in the least significant bits of {\tt s}.
 \endtab

\noindent
{\tt\#define bitset\_RotateRightLocal(s, b, r)}  ... 

\hide\code
#define bitset_RotateRightLocal(s, b, r) do { \
   uint64_t temp = (s) >> (r); \
   (s) <<= ((b) - (r));   (s) |= temp;   (s) &= bitset_MaskLow(b); \
   } while (0)
\endcode\endhide
 \tab  Rotates the {\tt b} lowest (least significant) bits of 
  the set {\tt s} by {\tt r} positions to the right.
  Here, {\tt s} is considered as a {\tt b}-bit number kept
  in the least significant bits of {\tt s}.
 \endtab


%%%%%%%%%%%%%%%%%%%%%%%%%%%%%%%%%%%%%%%%%%
\guisec{Prototypes}
\code

bitset_BitSet bitset_ReverseOrder (bitset_BitSet z, int b);
\endcode
\tab Reverses the order of the {\tt b} least significant bits of {\tt z}.
  Thus, if {\tt b}${}=4$ and {\tt z}${} = \dots 0011$, the returned value is $\dots 1100$.
	\pierre{The current implementation of this function is simple and very inefficient.
	 More efficient methods use lookup tables.   
	 See \url{https://graphics.stanford.edu/~seander/bithacks.html}.
	 I think this is implemented somewhere in TestU01.}
 \endtab
\code

void bitset_Write (char *desc, bitset_BitSet z, int b);
\endcode
  \tab  \pierre{Should return the string, not print it.  Then we can do whatever
	 we want with the string.}
  Prints the string {\tt desc} (which may be empty), then writes the {\tt b}
	least significant bits of {\tt z} considered as an unsigned binary number.
  This corresponds to the {\tt b} first elements of {\tt z}.
 \endtab
\code\hide
#endif
\endhide\endcode

\defmodule{bitvector}

This module offers facilities to store and manipulate bit vectors of arbitrary length.
The bit vectors are stored in arrays of 64-bit unsigned integers, and their length
is always rounded up to the nearest multiple of 64.
The bit indexation goes from left to right (which is unusual) and starts at 0.
This left-to-right ordering is because these vectors are used to construct
binary matrices (see \texttt{bitmatrix}),
for which the elements are usually ordered from left to right.
If the required length is not a multiple of 64, the unused bits are simply set to 0.


\bigskip\hrule

\code\hide
#ifndef BITVECTOR_H
#define BITVECTOR_H
\endhide
#include "gdef.h"
\endcode
\iffalse %%%
\code

#define bitvector_WL 64
\endcode
 \tab
Uses 64-bit words.   \pierre{Probably not needed, can be hardcoded to 64.}
 \endtab
\fi  %%%
\code

typedef struct{
   int n;
   uint64_t *vect;
} bitvector_vector;
\endcode
\tab
This structure contains a bit vector of 64\texttt{n} bits,
stored in {\tt n} blocks of 64 bits.
\hpierre{Do we also want to memorize the number of bits in the set?
  It may not be a multiple of 64.  But we also do not want to slow things down,
	so perhaps we may assume that the unused bits are always set to 0.
	Check what they do in Boost.}
Storage space for the {\tt bitvector\_vector} should be allocated via
{\tt bitvector\_allocate()}.
\endtab
\code

void bitvector_allocate (bitvector_vector *v, int b);
\endcode
 \tab
Allocates memory for a bit vector of {\tt b} bits.
There will be $n = \lceil b/64 \rceil$ blocks of 64 bits in the structure
and the value of $b$ is not saved.
\pierre{We may want to save it if needed, later.}
 \endtab
\code

void bitvector_free (bitvector_vector *v);
\endcode
 \tab
Releases the memory space used by {\tt bitvector\_vector} pointed by {\tt v}.
 \endtab
\code

void bitvector_display (bitvector_vector *v, int b);
\endcode
 \tab
Prints the first {\tt b} bits of the bit vector {\tt v}.
 \endtab
\code

void bitvector_copy (bitvector_vector *v1, bitvector_vector *v2);
  \endcode
 \tab
Copies the entire contents of {\tt v2} into {\tt v1}.
 \endtab
\code

void bitvector_copyPart (bitvector_vector *v1, bitvector_vector *v2, int b);
\endcode
 \tab
Copies the lowest {\tt b} bits of {\tt v2} into {\tt v1}.
 \endtab
\code

void bitvector_clearVector (bitvector_vector *v);
\endcode
 \tab
Resets all the bits of {\tt v} to zero.
 \endtab
\code

void bitvector_setBit (bitvector_vector *v, int b);
\endcode
 \tab
Sets bit {\tt b} of {\tt v} to 1.
 \endtab
\code

int bitvector_getBit (bitvector_vector *v, int b);
\endcode
 \tab
Returns the value of the {\tt b}-th bit of {\tt v}.
 \endtab
\code

void bitvector_clearBit (bitvector_vector *v, int b);
\endcode
 \tab
Sets bit {\tt b} of {\tt v} to 0.
 \endtab
\code

void bitvector_canonical (bitvector_vector *v, int t);
\endcode
 \tab
Sets {\tt v} to the $({\tt t}+1)$-th unit vector, with a 1 in position $t$.
That is, for {\tt t = 0}, we get te first unit vector $\be_1 = (1,0,0,\dots)$.
 \endtab
\code

void bitvector_setAllOnes (bitvector_vector *v);
\endcode
 \tab
Sets {\tt v} to the bit vector $(1, 1, 1, 1,\dots , 1)$.
 \endtab
\code

lebool bitvector_isZero (bitvector_vector *v);
\endcode
 \tab
Returns {\tt TRUE} if {\tt v} is the zero vector.  Returns {\tt FALSE} otherwise.
\endtab
\code

lebool bitvector_areEqual (bitvector_vector *v1, bitvector_vector *v2);
\endcode
\tab
Returns {\tt TRUE} if the bit vectors {\tt v1} and {\tt v2} are the same,
and {\tt FALSE} otherwise.
\endtab
\code

lebool bitvector_haveCommonBit (bitvector_vector *v1, bitvector_vector *v2);
\endcode
 \tab
Returns {\tt TRUE} if at least one bit set to 1 in {\tt v1} is also set to 1 in {\tt v2}
(they have at least one common bit).  Returns {\tt FALSE} otherwise.
 \endtab
\code

void bitvector_xor (bitvector_vector *v1, bitvector_vector *v2, bitvector_vector *v3);
\endcode
 \tab
Performs a bitwise exclusive-or of the contents of \texttt{v2} and \texttt{v3},
and puts the result in \texttt{v1}.
\endtab
\code

void bitvector_xor3 (bitvector_vector *v1, bitvector_vector *v2,
                     bitvector_vector *v3, bitvector_vector *v4);
\endcode
 \tab
Performs a bitwise exclusive-or of the contents of \texttt{v2}, \texttt{v3}, and \texttt{v4},
and puts the result in \texttt{v1}.
 \endtab
\code

void bitvector_xorSelf (bitvector_vector *v1, bitvector_vector *v2);
\endcode
 \tab
Performs a bitwise exclusive-or of the contents of \texttt{v1} and \texttt{v2},
and puts the result in \texttt{v1}.
 \endtab
\code

void bitvector_and (bitvector_vector *v1, bitvector_vector *v2, bitvector_vector *v3);
\endcode
 \tab
Performs a bitwise ``and'' of the contents of \texttt{v2} and \texttt{v3},
and puts the result in \texttt{v1}.
 \endtab
\code

void bitvector_andSelf (bitvector_vector *v1, bitvector_vector *v2);
\endcode
 \tab
Performs a bitwise ``and'' of the contents of \texttt{v1} and \texttt{v2},
and puts the result in \texttt{v1}.
 \endtab
\code

void bitvector_andMaskLow (bitvector_vector *v1, bitvector_vector *v2, int t);
\endcode
 \tab
Applies the mask comprised of {\tt t} ones followed by zeros to the bit vector
{\tt v2} and puts the result in {\tt v1}.
\endtab
\code

void bitvector_andInvMaskLow (bitvector_vector *v1, bitvector_vector *v2, int t);
\endcode
 \tab
Applies the mask comprised of {\tt t} zeros followed by ones to the bit vector
{\tt v2} and puts the result in {\tt v1}.
 \endtab
\code

void bitvector_leftShift (bitvector_vector *v1, bitvector_vector *v2, int b);
\endcode
 \tab
Performs a left shift of \texttt{v2} by \texttt{b} bits
and puts the result in \texttt{v1}.
 \endtab
\code

void bitvector_rightShift (bitvector_vector *v1, bitvector_vector *v2, int b);
\endcode
 \tab
Performs a right shift of \texttt{v2} by \texttt{b} bits
and puts the result in \texttt{v1}.
 \endtab
\code

void bitvector_leftShiftSelf (bitvector_vector *v, int b);
\endcode
 \tab
Performs a left shift of \texttt{v} by \texttt{b} bits
and puts the result in \texttt{v}.
 \endtab
\code

void bitvector_leftShift1Self (bitvector_vector *v);
\endcode
 \tab
Performs a left shift of \texttt{v} by one bit
and puts the result in \texttt{v}.
 \endtab
\code

void bitvector_rightShiftSelf (bitvector_vector *v, int b);
\endcode
 \tab
Performs a right shift of \texttt{v} by \texttt{b} bits
and puts the result in \texttt{v}.
 \endtab
\code

void bitvector_flip (bitvector_vector *v);
\endcode
 \tab
Flips (or toggle) the values of all the bits of {\tt v} (exchange 0 for 1, and vice-versa).
 \endtab
\code

void bitvector_setMaskLow (bitvector_vector *v, int t);
\endcode
 \tab
Fills {\tt v} with {\tt t} ones followed by zeros.
 \endtab
\code

void bitvector_setInvMaskLow (bitvector_vector *v, int t);
\endcode
 \tab
Fills {\tt v} with {\tt t} zeros followed by ones.
 \endtab
\code

void bitvector_fillRandomBits (bitvector_vector *v);
\endcode
 \tab
Fills the vector {\tt v} with random bits.
 \pierre{We need to provide a local implementation of this!!!  To be done later.}
 \endtab
\code\hide
#endif
\endhide\endcode


{\bf Other popular implementations of bit vectors:}

\mp{Simple tools for bit vectors in C:  \url{http://c-faq.com/misc/bitsets.html}.
Bits are counted from the left, starting at 0, as far as I understand.}

\mp{Another nice set of macros in C:
\url{https://github.com/iplinux/x11proto-trap/blob/master/xtrapbits.h}}

\mp{In \url{https://www.gnu.org/software/guile/docs/docs-1.8/guile-ref/Bit-Vectors.html},
bits are counted from left to right, starting at 0.
Different operations are available, such as Hamming weight of bit string.}

\mp{See \url{https://en.wikipedia.org/wiki/Bit_array} for bit vectors in various languages.
In \texttt{LatNet Builder} we use \texttt{dynamic\_bitset}, available in
the Boost C++ library:
\url{https://www.boost.org/doc/libs/1\_73\_0/libs/dynamic\_bitset/dynamic\_bitset.html}.}

\mp{In the dynamic bitsets of Boost, we  find the following:\small
``Each bit represents either the Boolean value true or false (1 or 0). To set a bit is to assign it 1. To clear or reset a bit is to assign it 0. To flip a bit is to change the value to 1 if it was 0 and to 0 if it was 1. Each bit has a non-negative position. A bitset x contains x.size() bits, with each bit assigned a unique position in the range [0,x.size()). The bit at position 0 is called the least significant bit and the bit at position size() - 1 is the most significant bit. When converting an instance of dynamic\_bitset to or from an unsigned long n, the bit at position i of the bitset has the same value
as {\tt (n >> i) \& 1}.''}

\defmodule{bitmatrix}

This module offers facilities to manipulate matrices of binary vectors
(from the module \texttt{bitvector}).
More specifically, each matrix has {\tt r} rows and {\tt t} columns,
and each entry is an {\tt b}-bit binary vector.
By taking {\tt t=1}, we get an ordinary binary matrix of {\tt r} rows and
{\tt b} columns of bits.
The more general form with ${\tt t} > 1$ is very convenient for the analysis of
equidistribution properties of $\FF_2$-linear random number generators
\cite{rLEC05a,rLEC09a,rPAN04t}.

\bigskip\hrule

\code\hide
#ifdef __cplusplus
extern "C" {
#endif
#ifndef BITMATRIX_H
#define BITMATRIX_H
\endhide
#include "gdef.h"
#include "bitvector.h"
\endcode

%%%%%%%%%%%%%%%%%%%%%%%%%%%%%%%%%%%%%%%%%%
%  \guisec{\bf \large Operations on binary matrices}
\code

typedef struct{
  bitvector_vector **rows;
  int r;
  int b;
  int t;
} bitmatrix_matrix;
\endcode
 \tab
This structure represents a matrix of {\tt r} rows and {\tt t} columns,
for which each entry is an {\tt b}-bit binary vector.
This gives a matrix with {\tt r} rows of {\tt t $\times$ b} bits each.
The variable {\tt rows} contains an array of {\tt r} arrays of {\tt t} {\tt b}-bit vectors.
Storage space for this array should be allocated via {\tt bitmatrix\_allocate()}.
\endtab
\code

void bitmatrix_allocate (bitmatrix_matrix* m, int r, int b, int t);
\endcode
\tab
Allocates space for a {\tt bitmatrix\_matrix} with {\tt r} rows and {\tt t} columns,
for which each entry is an {\tt b}-bit binary vector.
On each row, the function allocates space for {\tt t} bit vectors of {\tt b} bits each.
To allocate a simple $128 \times 128$ binary matrix, for example, one can invoke
{\tt bitmatrix\_allocate (m, 128, 128, 1)}.
%  in module \texttt{mecf}.
\endtab
\code

void bitmatrix_free (bitmatrix_matrix *m);
\endcode
 \tab
 Releases the space taken by the binary matrix {\tt m}.
 \endtab
\code

void bitmatrix_display (bitmatrix_matrix *m, int t, int l, int r);
\endcode
 \tab
Displays (print) the submatrix of {\tt *m} defined by the first {\tt r} rows and the
first {\tt l} bits of the first {\tt t} bit vectors.
 \endtab
 \code

int bitmatrix_hammingWeight (bitmatrix_matrix *m, int r);
\endcode
 \tab
 Returns the Hamming weight of row {\tt r} in matrix {\tt *m}.
 \endtab
 \code

int bitmatrix_weight (bitmatrix_matrix *m);
\endcode
 \tab
 Returns the sum of non-zero entries in matrix {\tt *m}.
 \endtab
 \code

void bitmatrix_copypart (bitmatrix_matrix *m1, bitmatrix_matrix *m2,
                         int r, int t);
\endcode
 \tab
Copies the first {\tt t} bit vectors of the first {\tt r} rows of
{\tt m2} into the first {\tt t} bit vectors of the first {\tt r} rows of {\tt m1}.
 \endtab
\code

void bitmatrix_copySpecial (bitmatrix_matrix *m1, bitmatrix_matrix *m2,
                            int nl, int *col, int t);
\endcode
 \tab
Copies the ({\tt t}-1) bit vectors indicated by the array {\tt col},
plus the first bit vector, on each of the {\tt nl} first rows of the matrix
{\tt m2} to the first {\tt t} bit vectors
on each of the first {\tt nl} rows of the matrix {\tt m1}.
 \endtab
\code

void bitmatrix_transpose (bitmatrix_matrix *m1, bitmatrix_matrix *m2,
                          int r, int t, int b);
\endcode
 \tab
Transposes the {\tt t} matrices of dimension {\tt r}$\times${\tt b}
found in {\tt m2} and puts the result in {\tt m1}.
 \endtab
\code

void bitmatrix_exchangeRows (bitmatrix_matrix *m, int i, int j);
\endcode
 \tab
Exchanges rows {\tt i} and {\tt j} in the binary matrix {\tt *m}.
 \endtab
\code

void bitmatrix_xorVect (bitmatrix_matrix *m, int r, int s, int min, int max);
\endcode
 \tab
Performs a exclusive-or between the {\tt s}-th and {\tt r}-th rows of {\tt m},
for the {\tt min}-th to the ({\tt max}-1)-th bit vectors only.
The result is put in row {\tt r} of {\tt m}.
%  {\tt m[r]} = {\tt m[r]} \verb1^1 {\tt m[s]}
 \endtab
\code

lebool bitmatrix_diagonalize (bitmatrix_matrix *m, int kg, int t, int l, int *gr);
\endcode
\tab
Diagonalizes the sub matrix of {\tt m} that consist of the first {\tt kg} rows
and the first {\tt l} bits of the first {\tt t} bit vectors on each row.
Returns {\tt TRUE} if the sub matrix is of full rank {\tt t}*{\tt l}.
In this case, the variable pointed by {\tt gr} remains unchanged.
Otherwise, returns FALSE and the variable pointed by {\tt gr} is changed
to the value of {\tt t} for which the function would have returned {\tt TRUE}.
 \endtab
\code

int bitmatrix_gaussianElimination (bitmatrix_matrix *m, int r, int l, int t);
\endcode
 \tab
Returns the rank of the submatrix of {\tt m} comprised of the first {\tt r} rows
and the first {\tt l} bits of the first {\tt t} bit vectors on each row.
 \endtab
\code

int bitmatrix_specialGaussianElimination (bitmatrix_matrix *m,
                                          int r, int l, int t, int *indices);
\endcode
 \tab
Returns the rank of the submatrix of {\tt m} formed by the first {\tt r} rows
and the first {\tt l} bits of the bit vectors indicated by the array {\tt indices}.
 \endtab
 \code

int bitmatrix_completeElimination (bitmatrix_matrix *m, int r, int l, int t);
\endcode
 \tab
This function tries to form an identity matrix by elimination. If it fails, it returns {\tt FALSE}.
Otherwise, it returns the rank of the submatrix of {\tt m} comprised of the first
{\tt r} rows and the first {\tt l} bits of the first {\tt t} bit vectors on each row.
 \endtab
\code

lebool bitmatrix_inverse (bitmatrix_matrix *minv, bitmatrix_matrix *m);
\endcode
\tab
Tries computing the inverse of {\tt m}, returns {\tt TRUE} if it succeeded, and puts the results in {\tt minv}.
Otherwise, returns {\tt FALSE}.
The sub matrices of the {\tt Matrix}s pointed by {\tt m} and {\tt minv} that
are considered are the ones composed of only the first column of the bit vectors.
 \endtab
\code

void bitmatrix_productByVector (bitvector_vector *v1, bitmatrix_matrix *m,
                                bitvector_vector *v2);
\endcode
\tab
Puts in {\tt v1} the product of {\bf m} by the vector {\tt v2}.
 \endtab
\code

void bitmatrix_product (bitmatrix_matrix *m1, bitmatrix_matrix *m2,
                        bitmatrix_matrix *m3);
\endcode
\tab
Compute the matrix product of {\tt m2} by {\tt m3} and puts the results in {\tt m1}.
Only the first submatrices are considered (i.e. the matrices composed of the first column of bit vectors).
\endtab
\code

void bitmatrix_power (bitmatrix_matrix *m1, bitmatrix_matrix *m2, int64_t e);
\endcode
\tab
Raises binary matrix {\tt m2} to the power {\tt e} and puts the results in {\tt m1}.
The exponent $e$ can be negative, in which case, the inverse of {\tt m2}
will be raised to the power $|e|$.
\endtab
\code

void bitmatrix_powerOfTwo (bitmatrix_matrix *m1, bitmatrix_matrix *m2,
                           unsigned int e);
\endcode
\tab
Raises binary matrix {\tt m2} to the power $2^{\tt e}$ and puts the results in {\tt m1}.
We get ${\tt m1} = ({\tt m2})^{2^{e}}$.
\endtab

\code\hide
#endif
#ifdef __cplusplus
}
#endif
\endhide\endcode

\defmodule{rngstream}

This module provides streams of random numbers constructed as proposed in \cite{rLEC02a},
using the combined multiple recursive generator {\tt Mrg32k3a} proposed in \cite{rLEC99b}
as a backbone generator.  This backbone generator has period length $\rho\approx 2^{191}$.
In those references, the generator was implemented in floating-point arithmetic
using ``\texttt{double}'' variables, whereas here it is implemented using 64-bit integers,
using the code proposed by S. Vigna.
The seed of the RNG, and the state of a stream at any given step,
are 6-dimensional vectors of 32-bit integers.
The default initial seed of the first stream is
$(12345, 12345, 12345, 12345, 12345, 12345)$,
and the seeds of the successive streams are spaced by $2^{127}$ steps.
Substreams are not implemented here.
Note that we prefer not to use an $\FF_2$-linear generator here because 
we sometimes use this module (e.g., in \texttt{bitvector} and in the 
\texttt{F2LinearGen} software) to generate random binary vectors and matrices. 


%%%%%%%%%%%%%%%%%%%%%%%
\bigskip\hrule

\code\hide
/* rngstream.h for ANSI C */
#ifdef __cplusplus
extern "C" {
#endif
#ifndef RNGSTREAM_H
#define RNGSTREAM_H
\endhide
#include "gdef.h"
#include "util.h"
#include "num.h"
#include <stdio.h>
#include <stdlib.h>

typedef struct rngstream_InfoState * rngstream;

struct rngstream_InfoState {
   int64_t Cg[6], Ig[6];
};
\endcode
 \tab
   The state of a stream from the present module.
   The arrays {\tt Ig} and {\tt Cg} contain the initial state
   and the current state, respectively.
 \endtab
\code

int rngstream_SetPackageSeed (int64_t seed[6]);
\endcode
  \tab  Sets the seed of the package {\tt rngstreams} to the
   six integers in the vector {\tt seed}.
   This will be the initial state of the first stream.
   If this procedure is not called, the default initial seed
   is $(12345, 12345, 12345, 12345, 12345, 12345)$.
   If it is called, the first 3 values of the seed must all be
   less than $m_1 = 4294967087$, and not all 0;
   and the last 3 values
   must all be less than $m_2 = 4294944443$, and not all 0.
   Returns 0 for valid seeds and exits otherwise.
 \endtab
\code

rngstream rngstream_CreateStream ();
\endcode
 \tab Creates and returns a new stream
   whose state variable is of type {\tt rngstream\_InfoState}.
   This procedure reserves space to keep the information relative to
   the {\tt rngstream}, initializes its seed $I_g$,
   and sets $B_g$ and $C_g$ equal to $I_g$.
   The seed $I_g$ is equal to the initial seed of the package given by
   {\tt rngstream\_SetPackageSeed} if this is the first stream created,
   otherwise it is $2^{127}$ steps ahead of that of the most recently created stream.
 \endtab
\code

void rngstream_DeleteStream (rngstream *g);
\endcode
 \tab Deletes the stream {\tt *g} created previously
  by {\tt rngstream\_CreateStream}, and recovers its memory.
  Otherwise, does nothing.
 \endtab
\code

void rngstream_ResetStartStream (rngstream g);
\endcode
 \tab Reinitializes the stream {\tt g} to its initial state:
   $C_g$ is set to $I_g$.
 \endtab
\code

int rngstream_SetSeed (rngstream g, int64_t seed[6]);
\endcode
 \tab  Sets the initial seed $I_g$ of stream {\tt g}
  to the vector {\tt seed}.  This vector must satisfy the same
  conditions as in {\tt rngstream\_SetPackageSeed}.
  The stream is then reset to this initial seed.
  The states and seeds of the other streams are not modified.
  As a result, after calling this procedure, the initial seeds
  of the streams are no longer spaced $2^{127}$ values apart.
  This function should be invoked only at the beginning of the program.
  Returns 0 for valid seeds and exits otherwise.
 \endtab
\code

void rngstream_GetState (rngstream g, int64_t seed[6]);
\endcode
 \tab Returns in {\tt seed[]} the current state $C_g$ of stream {\tt g}.
  This is convenient if we want to save the state for subsequent use.
 \endtab
\code

void rngstream_WriteState (rngstream g);
\endcode
 \tab Prints (to standard output) the current state of stream {\tt g}.
 \endtab
\code

double rngstream_RandU01 (rngstream g);
\endcode
 \tab Returns a (pseudo)random number from the uniform distribution
   over the interval $(0,1)$, using stream {\tt g},
   after advancing the state by one step.
   The returned number has 32 bits of precision in the sense that it is
   always a multiple of $1/(2^{32}-208)$.
 \endtab
\code

int rngstream_RandInt (rngstream g, int i, int j);
\endcode
 \tab Returns a (pseudo)random number from the discrete uniform
   distribution over the integers $\{i,i+1,\dots,j\}$, using stream {\tt g}.
   Makes one call to {\tt rngstream\_RandU01}.
 \endtab
\code\hide

#endif
#ifdef __cplusplus
}
#endif
\endhide
\endcode

\defmodule{chrono}

This module acts as an interface to the system clock to compute the
CPU time used by parts of a program.
Every variable of type {\tt chrono\_Chrono} acts as an independent 
{\em stopwatch}.  Several such stopwatchs can run at any given time.
An object of type {\tt chrono\_Chrono} must be declared 
for each of them.
The function {\tt chrono\_Init} resets the stopwatch to zero,
{\tt chrono\_Val\/} returns its current reading,
and {\tt chrono\_Write\/} writes this reading to the current output.
The returned value includes part of the execution time of the functions
from module {\tt chrono\/}.
The {\tt chrono\_TimeFormat} allows one to choose the kind of 
time units that are used.  
% When no longer needed, 
% the stopwatch can be deleted via {\tt chrono\_Delete}.

Below is an example of how the functions may be used.
A stopwatch named {\tt mytimer} is declared and created.
After 2.1 seconds of CPU time have been consumed, the stopwatch is read and
reset. Then, after an additional 330 seconds (or 5.5 minutes) of CPU time
the stopwatch is read again, printed to the output and deleted.
%
 \begin{verse}{\tt
  double t; \\
  chrono\_Chrono *mytimer = chrono\_Create (); \\
\hskip 1.0cm   \vdots 
\hskip 1.0cm  ({\em suppose 2.1 CPU seconds are used here}.)\\[6pt]
  t = chrono\_Val (mytimer, chrono\_sec); \qquad   /* Here, t = 2.1 */ \\
  chrono\_Init (mytimer); \\
\hskip 1.0cm  \vdots
\hskip 1.0cm ({\em suppose 330 CPU seconds are used here}.) \\[10pt]
  t = chrono\_Val (mytimer, chrono\_min); \qquad    /* Here, t = 5.5 */\\
  chrono\_Write (mytimer, chrono\_hms);  \qquad\ \  /* Prints: 00:05:30.00 */\\
  chrono\_Delete (mytimer);
 }\end{verse}

% When using this module, it is strongly recommended to leave the macro {\tt USE\_ANSI\_CLOCK} 
% in module {\tt gdef} undefined, otherwise the timer may %  {\tt clock}
% wrap around to negative values after about 36 minutes.
% {See  \url{https://stackoverflow.com/questions/39935820/accuracy-of-clock-function-in-c}}

On Linux-Unix systems, this module uses the POSIX
function {\tt times} to get the CPU time used by a program.
  On a Windows platform (when the macro \texttt{HAVE\_WINDOWS\_H} is defined),
  the Windows function \texttt{GetProcessTimes} is used instead.

\iffalse  %%%%%%
  Even though the ANSI/ISO macro {\tt CLOCKS\_PER\_SEC = 1000000} 
  is the number of clock ticks per second for the value
  returned by the {\tt clock} function (so this function returns the
  number of microseconds), on some systems where the 32-bit type {\tt long} 
  is used to measure time, the value returned by {\tt clock}
  wraps around to negative values after about 36 minutes.
  On some other systems where time is measured using the 32-bit type
  {\tt unsigned long}, the clock may wrap around to 0 after about 72 minutes.
\fi  %%%%%%

\code
\iffalse
/* chrono.h for ANSI C */

#ifdef __cplusplus
extern "C" {
#endif
#ifndef CHRONO_H
#define CHRONO_H 
#include "gdef.h"
\fi
\endcode

\newpage
%%%%%%%%%%%%%%%%%%%%%%%%%%%%%%%%%%%%%%%%%%
\guisec{Types}
\code

typedef struct {
   unsigned long microsec;
   unsigned long second;
   } chrono_Chrono;
\endcode
  \tab
   For every stopwatch needed, the user must declare a variable of
   this type and initialize it by calling {\tt chrono\_Create}.
  \endtab
\code

typedef enum {
   chrono_sec,
   chrono_min,
   chrono_hours, 
   chrono_days,
   chrono_hms
   } chrono_TimeFormat;
\endcode
 \tab
  Types of units in which the time on a {\tt chrono\_Chrono} can be 
  read or printed:
  in seconds ({\tt sec})), minutes ({\tt min}), hours ({\tt hour}), days
  ({\tt days}), or in the {\tt HH:MM:SS.xx} format, with hours, 
  minutes, seconds and hundreths of a second ({\tt hms}).
 \endtab


%%%%%%%%%%%%%%%%%%%%%%%%%%%%%%%%%%%%%%%%%%
\guisec{Timing functions}
\code

chrono_Chrono * chrono_Create (void);
\endcode
  \tab
   Creates and returns a stopwatch, after initializing it to zero. This
   function must be called for each new {\tt chrono\_Chrono} used.
   One may reinitializes it later by calling {\tt chrono\_Init}.
  \endtab
\code

void chrono_Delete (chrono_Chrono * C);
\endcode
  \tab
   Deletes the stopwatch {\tt C}.
  \endtab
\code

void chrono_Init (chrono_Chrono * C);
\endcode
  \tab
  Initializes the stopwatch {\tt C} to zero.
  \endtab
\code

double chrono_Val (chrono_Chrono * C, chrono_TimeFormat Unit);
\endcode
  \tab
  Returns the time used by the program since the last call to
  {\tt chrono\_Init(C)}. The parameter {\tt Unit} specifies the time unit.
  Restriction: {\tt Unit = chrono\_hms} is not allowed here.
%  it will cause an error.
  \endtab
\code

void chrono_Write (chrono_Chrono * C, chrono_TimeFormat Unit);
\endcode
 \tab
  Prints the CPU time used by the program since its last
  call to {\tt chrono\_Init(C)}.
  The parameter {\tt Unit} specifies the time unit.
 \endtab
\code
\iffalse
#endif
#ifdef __cplusplus
}
#endif
\fi\endcode

\defmodule{tcode}

\iffalse
 Le programme {\tt tcode} permet de produire du code compilable
\`a partir d'une documentation destin\'ee \`a TEX ou LATEX.
Il produit un fichier {\tt FOut} destin\'e \`a un compilateur tel que Modula-2
(ou autre), \`a partir d'un fichier {\tt FIn} re\c cu \`a l'entr\'ee.
Les noms de ces deux fichiers sont donn\'es par l'usager, avec l'extension,
lors de l'appel du programme.
N'appara\^\i tront dans le second fichier que les parties se trouvant entre
les d\'elimiteurs {\tt\bs{code}} et {\tt\bs{endcode}}.
Toutes les autres commandes TEX et LATEX, m\^eme \`a l'int\'erieur de ces
d\'elimiteurs, seront aussi enlev\'ees.
L'appel se fait sous la forme:
\fi


This small program extracts compilable code from a \TeX\ or \LaTeX\ document
that contains the documentation. 
It creates a file {\tt FOut} for a compiler like cc (or any other), starting
from a file {\tt FIn}. The names of these two files must be given by the user,
with appropriate extensions, when calling the program.
The two file names (with the extensions) must be different.

Only the text included between the {\tt\bs{code}} and 
{\tt\bs{endcode}} delimiters will appear in the second file. Only the following
\LaTeX\  commands are allowed between {\tt\bs{code}} and {\tt\bs{endcode}}:

\begin{verse}
 {\tt\bs{hide}}, {\tt\bs{endhide}}, {\tt\bs{iffalse}}, {\tt\bs{fi}},
  {\tt\bs{smallcode}},   {\tt\bs{smallc}}.
\end{verse}

Everything else between
{\tt\bs{code}} and {\tt\bs{endcode}} must be legal code in the 
output file, apart from two exceptions: the \TeX\  command
 {\tt\bs{def}\bs{code}}, defining  {\tt\bs{code}} will not start a region
 of valid code, nor will  {\tt\bs{code}} appearing on a line after a
 \TeX\  comment character {\tt\%}.

If one wants code to appear in the compilable file, but be invisible in the 
\LaTeX\ output file (e.g., {\tt .pdf} or {\tt .dvi}), 
%  or  file obtained from processing the {\tt tex} file with \LaTeX, 
it suffices to put this code between the delimiters
{\tt\bs{hide}} and {\tt\bs{endhide}}, or between the delimiters
{\tt\bs{iffalse}} and {\tt\bs{fi}}.

The program is called as:

\begin {center}\tt
  tcode \ $\langle$FIn$\rangle$ \ $\langle$FOut$\rangle$
\end {center}

\paragraph{Examples:}
The following command extracts the {\it C} code  from the \LaTeX\ file {\tt chrono.tex},
 and place it in the header file {\tt chrono.h}:

\begin {center}\tt
  tcode \ chrono.tex \ chrono.h
\end {center}

To extract {\it Java} code  from the \LaTeX\  file {\tt Event.tex},
and place it in the file {\tt Event.java}, one would use:

\begin {center}\tt
  tcode \ Event.tex \ Event.java
\end {center}


\fi


\clearpage\input gdef.tex
\clearpage\input util.tex
\clearpage\input num.tex
\clearpage\input num2.tex
\clearpage\input mystr.tex
\clearpage\input addstr.tex
\clearpage\input tables.tex
\clearpage\input bitset.tex
\clearpage\input bitvector.tex
\clearpage\input bitmatrix.tex
\clearpage\input rngstream.tex
\clearpage\input chrono.tex
\clearpage\input tcode.tex
\clearpage\input extranotes.tex

\clearpage
\bibliographystyle {plain}
\bibliography{stat,random,simul,math,ift}
\end{document}
