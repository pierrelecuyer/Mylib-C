\defmodule{rngstream}

This module provides streams of random numbers constructed as proposed in \cite{rLEC02a},
using the combined multiple recursive generator {\tt Mrg32k3a} proposed in \cite{rLEC99b}
as a backbone generator.  This backbone generator has period length $\rho\approx 2^{191}$.
In those references, the generator was implemented in floating-point arithmetic
using ``\texttt{double}'' variables, whereas here it is implemented using 64-bit integers,
using the code proposed by S. Vigna.
The seed of the RNG, and the state of a stream at any given step,
are 6-dimensional vectors of 32-bit integers.
The default initial seed of the first stream is
$(12345, 12345, 12345, 12345, 12345, 12345)$,
and the seeds of the successive streams are spaced by $2^{127}$ steps.
Substreams are not implemented here.


%%%%%%%%%%%%%%%%%%%%%%%
\bigskip\hrule

\code\hide
/* rngstream.h for ANSI C */
#ifndef RNGSTREAM_H
#define RNGSTREAM_H
\endhide
#include "gdef.h"
#include "util.h"
#include "num.h"
#include <stdio.h>
#include <stdlib.h>

typedef struct rngstream_InfoState * rngstream;

struct rngstream_InfoState {
   int64_t Cg[6], Ig[6];
};
\endcode
 \tab
   The state of a stream from the present module.
   The arrays {\tt Ig} and {\tt Cg} contain the initial state
   and the current state, respectively.
 \endtab
\code

int rngstream_SetPackageSeed (int64_t seed[6]);
\endcode
  \tab  Sets the seed of the package {\tt rngstreams} to the
   six integers in the vector {\tt seed}.
   This will be the initial state of the first stream.
   If this procedure is not called, the default initial seed
   is $(12345, 12345, 12345, 12345, 12345, 12345)$.
   If it is called, the first 3 values of the seed must all be
   less than $m_1 = 4294967087$, and not all 0;
   and the last 3 values
   must all be less than $m_2 = 4294944443$, and not all 0.
   Returns 0 for valid seeds and exits otherwise.
 \endtab
\code

rngstream rngstream_CreateStream ();
\endcode
 \tab Creates and returns a new stream
   whose state variable is of type {\tt rngstream\_InfoState}.
   This procedure reserves space to keep the information relative to
   the {\tt rngstream}, initializes its seed $I_g$,
   sets $B_g$ and $C_g$ equal to $I_g$, sets its antithetic and precision
   switches to 0.
   The seed $I_g$ is equal to the initial seed of the package given by
   {\tt rngstream\_SetPackageSeed} if this is the first stream created,
   otherwise it is $Z$ steps ahead of that of the most recently
   created stream.
 \endtab
\code

void rngstream_DeleteStream (rngstream *p);
\endcode
 \tab Deletes the stream {\tt *pg} created previously
  by {\tt rngstream\_CreateStream}, and recovers its memory.
  Otherwise, does nothing.
 \endtab
\code

void rngstream_ResetStartStream (rngstream g);
\endcode
 \tab Reinitializes the stream {\tt g} to its initial state:
   $C_g$ is set to $I_g$.
 \endtab
\code

int rngstream_SetSeed (rngstream g, int64_t seed[6]);
\endcode
 \tab  Sets the initial seed $I_g$ of stream {\tt g}
  to the vector {\tt seed}.  This vector must satisfy the same
  conditions as in {\tt rngstream\_SetPackageSeed}.
  The stream is then reset to this initial seed.
  The states and seeds of the other streams are not modified.
  As a result, after calling this procedure, the initial seeds
  of the streams are no longer spaced $Z$ values apart.
  This function should be invoked only at the beginning of the program.
  Returns 0 for valid seeds and exits otherwise.
 \endtab
\code

void rngstream_GetState (rngstream g, int64_t seed[6]);
\endcode
 \tab Returns in {\tt seed[]} the current state $C_g$ of stream {\tt g}.
  This is convenient if we want to save the state for subsequent use.
 \endtab
\code

void rngstream_WriteState (rngstream g);
\endcode
 \tab Prints (to standard output) the current state of stream {\tt g}.
 \endtab
\code

double rngstream_RandU01 (rngstream g);
\endcode
 \tab Returns a (pseudo)random number from the uniform distribution
   over the interval $(0,1)$, using stream {\tt g},
   after advancing the state by one step.
   The returned number has 32 bits of precision in the sense that it is
   always a multiple of $1/(2^{32}-208)$.
 \endtab
\code

int rngstream_RandInt (rngstream g, int i, int j);
\endcode
 \tab Returns a (pseudo)random number from the discrete uniform
   distribution over the integers $\{i,i+1,\dots,j\}$, using stream {\tt g}.
   Makes one call to {\tt rngstream\_RandU01}.
 \endtab
\code\hide

#endif
\endhide
\endcode
