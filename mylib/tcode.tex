\defmodule{tcode}

\iffalse
 Le programme {\tt tcode} permet de produire du code compilable
\`a partir d'une documentation destin\'ee \`a TEX ou LATEX.
Il produit un fichier {\tt FOut} destin\'e \`a un compilateur tel que Modula-2
(ou autre), \`a partir d'un fichier {\tt FIn} re\c cu \`a l'entr\'ee.
Les noms de ces deux fichiers sont donn\'es par l'usager, avec l'extension,
lors de l'appel du programme.
N'appara\^\i tront dans le second fichier que les parties se trouvant entre
les d\'elimiteurs {\tt\bs{code}} et {\tt\bs{endcode}}.
Toutes les autres commandes TEX et LATEX, m\^eme \`a l'int\'erieur de ces
d\'elimiteurs, seront aussi enlev\'ees.
L'appel se fait sous la forme:
\fi


This small program extracts compilable code from a \TeX\ or \LaTeX\ document
that contains the documentation. 
It creates a file {\tt FOut} for a compiler like cc (or any other), starting
from a file {\tt FIn}. The names of these two files must be given by the user,
with appropriate extensions, when calling the program.
The two file names (with the extensions) must be different.

Only the text included between the {\tt\bs{code}} and 
{\tt\bs{endcode}} delimiters will appear in the second file. Only the following
\LaTeX\  commands are allowed between {\tt\bs{code}} and {\tt\bs{endcode}}:

\begin{verse}
 {\tt\bs{hide}}, {\tt\bs{endhide}}, {\tt\bs{iffalse}}, {\tt\bs{fi}},
  {\tt\bs{smallcode}},   {\tt\bs{smallc}}.
\end{verse}

Everything else between
{\tt\bs{code}} and {\tt\bs{endcode}} must be legal code in the 
output file, apart from two exceptions: the \TeX\  command
 {\tt\bs{def}\bs{code}}, defining  {\tt\bs{code}} will not start a region
 of valid code, nor will  {\tt\bs{code}} appearing on a line after a
 \TeX\  comment character {\tt\%}.

If one wants code to appear in the compilable file, but be invisible in the 
\LaTeX\ output file (e.g., {\tt .pdf} or {\tt .dvi}), 
%  or  file obtained from processing the {\tt tex} file with \LaTeX, 
it suffices to put this code between the delimiters
{\tt\bs{hide}} and {\tt\bs{endhide}}, or between the delimiters
{\tt\bs{iffalse}} and {\tt\bs{fi}}.

The program is called as:

\begin {center}\tt
  tcode \ $\langle$FIn$\rangle$ \ $\langle$FOut$\rangle$
\end {center}

\paragraph{Examples:}
The following command extracts the {\it C} code  from the \LaTeX\ file {\tt chrono.tex},
 and place it in the header file {\tt chrono.h}:

\begin {center}\tt
  tcode \ chrono.tex \ chrono.h
\end {center}

To extract {\it Java} code  from the \LaTeX\  file {\tt Event.tex},
and place it in the file {\tt Event.java}, one would use:

\begin {center}\tt
  tcode \ Event.tex \ Event.java
\end {center}

