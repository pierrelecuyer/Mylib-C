\defmodule {tables}

This module provides an implementation of variable-sized arrays (matrices),
and procedures to manipulate them.
The advantage is that the size of the array needs not be known
at compile time; it can be specified only during the program execution.
There are also procedures to sort arrays,  to
print  arrays in different formats,
and a few tools for hashing tables.
The functions {\tt tables\_CreateMatrix...} and
{\tt tables\_DeleteMatrix...} manage memory allocation for
these dynamic matrices.

As an illustration, the following piece of code declares and creates
a $100\times 500$ table of floating point numbers, assigns a value
to one table entry, and eventually deletes the table:
  \begin{verse}{\tt
    double ** T;\\
    T = tables\_CreateMatrix\_double (100, 500);\\
    T[3][7] = 1.234;\\
    \dots \\
    tables\_DeleteMatrix\_double (\&T);
  }\end{verse}

%%%%%%%%%%%%%
\bigskip\hrule

\code\hide
/* tables.h for ANSI C */
#ifndef TABLES_H
#define TABLES_H
\endhide
#include "gdef.h"
\endcode

%%%%%%%%%%%%%%%%%%%%%%%%%%%%%%%%%%%%%%%%%%
\guisec{Printing styles}
\code

typedef enum {
   tables_Plain,
   tables_Mathematica,
   tables_Matlab
   } tables_StyleType;
\endcode
  \tab Printing styles for matrices.
  \endtab

%%%%%%%%%%%%%%%%%%%%%%%%%%%%%%%%%%%%%%%%%%
\guisec{Functions to create, delete, sort, and print tables}
\code

long ** tables_CreateMatrix_long  (int m, int n);
unsigned long ** tables_CreateMatrix_ulong (int m, int n);
int64_t ** tables_CreateMatrix_int64  (int m, int n);
uint64_t long ** tables_CreateMatrix_uint64 (int m, int n);
double ** tables_CreateMatrix_double  (int m, int n);
\endcode
  \tab Allocates contiguous memory for a dynamic 
  matrix of {\tt m} rows and {\tt n} columns. Returns the base
  address of the allocated space.
  \endtab
\code

void tables_DeleteMatrix_long  (long *** mat);
void tables_DeleteMatrix_ulong (unsigned long *** mat);
void tables_DeleteMatrix_int64  (int64_t *** mat);
void tables_DeleteMatrix_uint64 (uint64_t *** mat);
void tables_DeleteMatrix_double  (double *** mat);
\endcode
  \tab Releases the memory used by the matrix {\tt mat}
  (see {\tt tables\_CreateMatrix}) passed by reference; i.e., using the {\tt \&} symbol. 
  Then, {\tt mat} is set to {\tt NULL}.
  \endtab
\code

void tables_CopyTab_long (long mat1[], long mat2[], int n1, int n2);
void tables_CopyTab_int64 (long mat1[], long mat2[], int n1, int n2);
void tables_CopyTab_uint64 (long mat1[], long mat2[], int n1, int n2);
void tables_CopyTab_double (double mat1[], double mat2[], int n1, int n2);
\endcode
  \tab Copies {\tt mat1[n1..n2]} in {\tt mat2[n1..n2]}.
	\pierre{Oups... I think in C, the copy order is usually the other way!  Check this. }
  \endtab
\code

void tables_QuickSort_long (long mat[], int n1, int n2);
void tables_QuickSort_int64 (int64_t mat[], int n1, int n2);
void tables_QuickSort_uint64 (uint64_t mat[], int n1, int n2);
void tables_QuickSort_double (double mat[], int n1, int n2);
\endcode
 \tab Sort the tables {\tt mat[n1..n2]} in increasing order.
 \endtab
\code

void tables_WriteTab_long (long mat[], int n1, int n2, int k, int p, 
                           char desc[]);
void tables_WriteTab_int64 (int64_t mat[], int n1, int n2, int k, int p, 
                            char desc[]);
void tables_WriteTab_uint64 (uint64_t mat[], int n1, int n2, int k, int p, 
                             char desc[]);
\endcode
 \tab  Write the elements {\tt n1} to {\tt n2} of table {\tt mat},
  {\tt k} per line, {\tt p} positions per element.
  If  {\tt k} = 1, the index will also be printed. {\tt desc}
  contains a description of the table.
 \endtab
\code

void tables_WriteTab_double (double mat[], int n1, int n2, int k, 
                             int p1, int p2, int p3, char desc[]);
\endcode
 \tab  Writes the elements {\tt n1} to {\tt n2} of table {\tt mat},
  {\tt k} per line, with at least {\tt p1} positions per element,
  {\tt p2} digits after the decimal point, and at least  {\tt p3} significant digits.
   If {\tt k} = 1, the index
  will also be printed. {\tt desc} contains a description of the table.
 \endtab\code

void tables_WriteSubmatrix_double (double** mat, int i1, int i2, int j1, int j2,
                             int w, int p, tables_StyleType style, char name[]);
\endcode
 \tab Writes the submatrix with lines 
   {\tt i1} $\le i \le $ {\tt i2} and columns 
   {\tt j1} $\le j \le $ {\tt j2} of the matrix {\tt mat} with format
   {\tt style}. The elements are printed in {\tt w}
   positions with a precision of {\tt p} digits. {\tt name} is
   an identifier for the submatrix.
  
   For {\tt Matlab}, the file containing the printed submatrix should have 
	 the extension {\tt .m}.
   For example, if it is named {\tt poil.m}, it will be accessed by the
   simple call {\tt poil} in {\tt Matlab}.
   For {\tt Mathematica}, if the file is named {\tt poil},
   it will be read using {\tt << poil;}.
 \endtab
\code

void tables_WriteSubmatrix_long (long** mat, int i1, int i2, int j1, int j2, 
                                 int w, tables_StyleType style, char name[]);
void tables_WriteSubmatrix_int64 (int64** mat, int i1, int i2, int j1, int j2,
                                  int w, tables_StyleType style, char name[]);
void tables_WriteSubmatrix_uint64 (uint64** mat, int i1, int i2, int j1, int j2, 
                                   int w, tables_StyleType style, char name[]);
\endcode
 \tab Similar to {\tt tables\_WriteMatrix\_double}.
 \endtab
\code

int64_t tables_HashPrime (int64_t n, double load);
\endcode
  \tab Returns a prime number $p$ to be used as the size 
   (the number of elements) of a hashing table.
   $p$ will be such that the load factor $n/p$ do not exceed {\tt load}.
   If {\tt load} is small, an important part of the table will be unused; that
   will accelerate searches and insertions.
   This function uses a small sequence of prime numbers; the real load factor
   may be significantly smaller than {\tt load} because only a limited
   number of prime numbers are in the table. In case of failure, returns $-1$.
 \endtab
\code

long tables_HashPrime32 (long n, double load);
\endcode
  \tab Same as \texttt{tables\_HashPrime}, but with 32-bit integers.
 \endtab
\code\hide
#endif
\endhide
\endcode
