\defmodule{bitmatrix}

This module offers facilities to manipulate matrices of binary vectors
(from the module \texttt{bitvector}).
More specifically, each matrix has {\tt r} rows and {\tt t} columns,
and each entry is an {\tt b}-bit binary vector.
By taking {\tt t=1}, we get an ordinary binary matrix of {\tt r} rows and
{\tt b} columns of bits.
The more general form with ${\tt t} > 1$ is very convenient for the analysis of
equidistribution properties of $\FF_2$-linear random number generators
\cite{rLEC05a,rLEC09a,rPAN04t}.

\bigskip\hrule

\code\hide
#ifndef BITMATRIX_H
#define BITMATRIX_H
\endhide
#include "gdef.h"
#include "bitvector.h"
\endcode

%%%%%%%%%%%%%%%%%%%%%%%%%%%%%%%%%%%%%%%%%%
%  \guisec{\bf \large Operations on binary matrices}
\code

typedef struct{
  bitvector_vector **rows;
  int r;
  int b;
  int t;
} bitmatrix_matrix;
\endcode
 \tab
This structure represents a matrix of {\tt r} rows and {\tt t} columns,
for which each entry is an {\tt b}-bit binary vector.
This gives a matrix with {\tt r} rows of {\tt t $\times$ b} bits each.
The variable {\tt rows} contains an array of {\tt r} arrays of {\tt t} {\tt b}-bit vectors.
Storage space for this array should be allocated via {\tt bitmatrix\_allocate()}.
\endtab
\code

void bitmatrix_allocate (bitmatrix_matrix* m, int r, int b, int t);
\endcode
\tab
Allocates space for a {\tt bitmatrix\_matrix} with {\tt r} rows and {\tt t} columns,
for which each entry is an {\tt b}-bit binary vector.
On each row, the function allocates space for {\tt t} bit vectors of {\tt b} bits each.
To allocate a simple $128 \times 128$ binary matrix, for example, one can invoke
{\tt bitmatrix\_allocate (m, 128, 128, 1)}.
%  in module \texttt{mecf}.
\endtab
\code

void bitmatrix_free (bitmatrix_matrix *m);
\endcode
 \tab
 Releases the space taken by the binary matrix {\tt m}.
 \endtab
\code

void bitmatrix_display (bitmatrix_matrix *m, int t, int l, int r);
\endcode
 \tab
Displays (print) the submatrix of {\tt *m} defined by the first {\tt r} rows and the
first {\tt l} bits of the first {\tt t} bit vectors.
 \endtab
 \code

int bitmatrix_hammingWeight (bitmatrix_matrix *m, int r);
\endcode
 \tab
 Returns the Hamming weight of row {\tt r} in matrix {\tt *m}.
 \endtab
 \code

int bitmatrix_weight (bitmatrix_matrix *m);
\endcode
 \tab
 Returns the sum of non-zero entries in matrix {\tt *m}.
 \endtab
 \code

void bitmatrix_copypart (bitmatrix_matrix *m1, bitmatrix_matrix *m2,
                         int r, int t);
\endcode
 \tab
Copies the first {\tt t} bit vectors of the first {\tt r} rows of
{\tt m2} into the first {\tt t} bit vectors of the first {\tt r} rows of {\tt m1}.
 \endtab
\code

void bitmatrix_copySpecial (bitmatrix_matrix *m1, bitmatrix_matrix *m2,
                            int nl, int *col, int t);
\endcode
 \tab
Copies the ({\tt t}-1) bit vectors indicated by the array {\tt col},
plus the first bit vector, on each of the {\tt nl} first rows of the matrix
{\tt m2} to the first {\tt t} bit vectors
on each of the first {\tt nl} rows of the matrix {\tt m1}.
 \endtab
\code

void bitmatrix_transpose (bitmatrix_matrix *m1, bitmatrix_matrix *m2,
                          int r, int t, int b);
\endcode
 \tab
Transposes the {\tt t} matrices of dimension {\tt r}$\times${\tt b}
found in {\tt m2} and puts the result in {\tt m1}.
 \endtab
\code

void bitmatrix_exchangeRows (bitmatrix_matrix *m, int i, int j);
\endcode
 \tab
Exchanges rows {\tt i} and {\tt j} in the binary matrix {\tt *m}.
 \endtab
\code

void bitmatrix_xorVect (bitmatrix_matrix *m, int r, int s, int min, int max);
\endcode
 \tab
Performs a exclusive-or between the {\tt s}-th and {\tt r}-th rows of {\tt m},
for the {\tt min}-th to the ({\tt max}-1)-th bit vectors only.
The result is put in row {\tt r} of {\tt m}.
%  {\tt m[r]} = {\tt m[r]} \verb1^1 {\tt m[s]}
 \endtab
\code

lebool bitmatrix_diagonalize (bitmatrix_matrix *m, int kg, int t, int l, int *gr);
\endcode
\tab
Diagonalizes the sub matrix of {\tt m} that consist of the first {\tt kg} rows
and the first {\tt l} bits of the first {\tt t} bit vectors on each row.
Returns {\tt TRUE} if the sub matrix is of full rank {\tt t}*{\tt l}.
In this case, the variable pointed by {\tt gr} remains unchanged.
Otherwise, returns FALSE and the variable pointed by {\tt gr} is changed
to the value of {\tt t} for which the function would have returned {\tt TRUE}.
 \endtab
\code

int bitmatrix_gaussianElimination (bitmatrix_matrix *m, int r, int l, int t);
\endcode
 \tab
Returns the rank of the submatrix of {\tt m} comprised of the first {\tt r} rows
and the first {\tt l} bits of the first {\tt t} bit vectors on each row.
 \endtab
\code

int bitmatrix_specialGaussianElimination (bitmatrix_matrix *m,
                                          int r, int l, int t, int *indices);
\endcode
 \tab
Returns the rank of the submatrix of {\tt m} formed by the first {\tt r} rows
and the first {\tt l} bits of the bit vectors indicated by the array {\tt indices}.
 \endtab
 \code

int bitmatrix_completeElimination (bitmatrix_matrix *m, int r, int l, int t);
\endcode
 \tab
This function tries to form an identity matrix by elimination. If it fails, it returns {\tt FALSE}.
Otherwise, it returns the rank of the submatrix of {\tt m} comprised of the first
{\tt r} rows and the first {\tt l} bits of the first {\tt t} bit vectors on each row.
 \endtab
\code

lebool bitmatrix_inverse (bitmatrix_matrix *minv, bitmatrix_matrix *m);
\endcode
\tab
Tries computing the inverse of {\tt m}, returns {\tt TRUE} if it succeeded, and puts the results in {\tt minv}.
Otherwise, returns {\tt FALSE}.
The sub matrices of the {\tt Matrix}s pointed by {\tt m} and {\tt minv} that
are considered are the ones composed of only the first column of the bit vectors.
 \endtab
\code

void bitmatrix_productByVector (bitvector_vector *v1, bitmatrix_matrix *m,
                                bitvector_vector *v2);
\endcode
\tab
Puts in {\tt v1} the product of {\bf m} by the vector {\tt v2}.
 \endtab
\code

void bitmatrix_product (bitmatrix_matrix *m1, bitmatrix_matrix *m2,
                        bitmatrix_matrix *m3);
\endcode
\tab
Compute the matrix product of {\tt m2} by {\tt m3} and puts the results in {\tt m1}.
Only the first submatrices are considered (i.e. the matrices composed of the first column of bit vectors).
\endtab
\code

void bitmatrix_power (bitmatrix_matrix *m1, bitmatrix_matrix *m2, int64_t e);
\endcode
\tab
Raises binary matrix {\tt m2} to the power {\tt e} and puts the results in {\tt m1}.
The exponent $e$ can be negative, in which case, the inverse of {\tt m2}
will be raised to the power $|e|$.
\endtab
\code

void bitmatrix_powerOfTwo (bitmatrix_matrix *m1, bitmatrix_matrix *m2,
                           unsigned int e);
\endcode
\tab
Raises binary matrix {\tt m2} to the power $2^{\tt e}$ and puts the results in {\tt m1}.
We get ${\tt m1} = ({\tt m2})^{2^{e}}$.
\endtab

\code\hide
#endif
\endhide\endcode
