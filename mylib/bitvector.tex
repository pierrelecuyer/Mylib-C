\defmodule{bitvector}

This module offers facilities to store and manipulate bit vectors of arbitrary length.
The bit vectors are stored in arrays of 64-bit unsigned integers, and their length
is always rounded up to the nearest multiple of 64.
The bit indexation goes from left to right and starts at 0.
This left-to-right ordering is because these vectors are used to construct
binary matrices (see \texttt{bitmatrix}),
for which the elements are usually ordered from left to right.
If the required length is not a multiple of 64, the unused bits are simply set to 0.


\bigskip\hrule

\code\hide
#ifndef BITVECTOR_H
#define BITVECTOR_H
\endhide
#include "gdef.h"
\endcode
\iffalse %%%
\code

#define bitvector_WL 64
\endcode
 \tab
Uses 64-bit words.   \pierre{Probably not needed, can be hardcoded to 64.}
 \endtab
\fi  %%%
\code

typedef struct{
   int n;
   uint64_t *vect;
} bitvector_vector;
\endcode
\tab
This structure contains a bit vector of 64\texttt{n} bits,
stored in {\tt n} blocks of 64 bits.
\hpierre{Do we also want to memorize the number of bits in the set?
  It may not be a multiple of 64.  But we also do not want to slow things down,
	so perhaps we may assume that the unused bits are always set to 0.
	Check what they do in Boost.}
Storage space for the {\tt bitvector\_vector} should be allocated via
{\tt bitvector\_allocate()}.
\endtab
\code

void bitvector_allocate (bitvector_vector *v, int b);
\endcode
 \tab
Allocates memory for a bit vector of {\tt b} bits.
There will be $n = \lceil b/64 \rceil$ blocks of 64 bits in the structure
and the value of $b$ is not saved.
\hpierre{We may want to save it if needed, later.}
 \endtab
\code

void bitvector_free (bitvector_vector *v);
\endcode
 \tab
Releases the memory space used by {\tt bitvector\_vector} pointed by {\tt v}.
 \endtab
\code

void bitvector_display (bitvector_vector *v, int b);
\endcode
 \tab
Prints the first {\tt b} bits of the bit vector {\tt v}.
 \endtab
\code

void bitvector_copy (bitvector_vector *v1, bitvector_vector *v2);
  \endcode
 \tab
Copies the entire contents of {\tt v2} into {\tt v1}.
 \endtab
\code

void bitvector_copyPart (bitvector_vector *v1, bitvector_vector *v2, int b);
\endcode
 \tab
Copies the lowest {\tt b} bits of {\tt v2} into {\tt v1}.
 \endtab
\code

void bitvector_clearVector (bitvector_vector *v);
\endcode
 \tab
Resets all the bits of {\tt v} to zero.
 \endtab
\code

void bitvector_setBit (bitvector_vector *v, int b);
\endcode
 \tab
Sets bit {\tt b} of {\tt v} to 1.
 \endtab
\code

int bitvector_getBit (bitvector_vector *v, int b);
\endcode
 \tab
Returns the value of the {\tt b}-th bit of {\tt v}.
 \endtab
\code

void bitvector_clearBit (bitvector_vector *v, int b);
\endcode
 \tab
Sets bit {\tt b} of {\tt v} to 0.
 \endtab
\code

int bitvector_hammingWeight (bitvector_vector *v);
\endcode
 \tab
Returns Hamming weight of {\tt v}.
 \endtab
\code

void bitvector_canonical (bitvector_vector *v, int t);
\endcode
 \tab
Sets {\tt v} to the $({\tt t}+1)$-th unit vector, with a 1 in position $t$.
That is, for {\tt t = 0}, we get te first unit vector $\be_1 = (1,0,0,\dots)$.
 \endtab
\code

void bitvector_setAllOnes (bitvector_vector *v);
\endcode
 \tab
Sets {\tt v} to the bit vector $(1, 1, 1, 1,\dots , 1)$.
 \endtab
\code

lebool bitvector_isZero (bitvector_vector *v);
\endcode
 \tab
Returns {\tt TRUE} if {\tt v} is the zero vector.  Returns {\tt FALSE} otherwise.
\endtab
\code

lebool bitvector_areEqual (bitvector_vector *v1, bitvector_vector *v2);
\endcode
\tab
Returns {\tt TRUE} if the bit vectors {\tt v1} and {\tt v2} are the same,
and {\tt FALSE} otherwise.
\endtab
\code

lebool bitvector_haveCommonBit (bitvector_vector *v1, bitvector_vector *v2);
\endcode
 \tab
Returns {\tt TRUE} if at least one bit set to 1 in {\tt v1} is also set to 1 in {\tt v2}
(they have at least one common bit).  Returns {\tt FALSE} otherwise.
 \endtab
\code

void bitvector_xor (bitvector_vector *v1, bitvector_vector *v2, 
                    bitvector_vector *v3);
\endcode
 \tab
Performs a bitwise exclusive-or of the contents of \texttt{v2} and \texttt{v3},
and puts the result in \texttt{v1}.
\endtab
\code

void bitvector_xor3 (bitvector_vector *v1, bitvector_vector *v2,
                     bitvector_vector *v3, bitvector_vector *v4);
\endcode
 \tab
Performs a bitwise exclusive-or of the contents of \texttt{v2}, \texttt{v3}, and \texttt{v4},
and puts the result in \texttt{v1}.
 \endtab
\code

void bitvector_xorSelf (bitvector_vector *v1, bitvector_vector *v2);
\endcode
 \tab
Performs a bitwise exclusive-or of the contents of \texttt{v1} and \texttt{v2},
and puts the result in \texttt{v1}.
 \endtab
\code

void bitvector_and (bitvector_vector *v1, bitvector_vector *v2, 
                    bitvector_vector *v3);
\endcode
 \tab
Performs a bitwise ``and'' of the contents of \texttt{v2} and \texttt{v3},
and puts the result in \texttt{v1}.
 \endtab
\code

void bitvector_andSelf (bitvector_vector *v1, bitvector_vector *v2);
\endcode
 \tab
Performs a bitwise ``and'' of the contents of \texttt{v1} and \texttt{v2},
and puts the result in \texttt{v1}.
 \endtab
\code

void bitvector_andMaskLow (bitvector_vector *v1, bitvector_vector *v2, int t);
\endcode
 \tab
Applies the mask comprised of {\tt t} ones followed by zeros to the bit vector
{\tt v2} and puts the result in {\tt v1}.
\endtab
\code

void bitvector_andInvMaskLow (bitvector_vector *v1, bitvector_vector *v2, int t);
\endcode
 \tab
Applies the mask comprised of {\tt t} zeros followed by ones to the bit vector
{\tt v2} and puts the result in {\tt v1}.
 \endtab
\code

void bitvector_leftShift (bitvector_vector *v1, bitvector_vector *v2, int b);
\endcode
 \tab
Performs a left shift of \texttt{v2} by \texttt{b} bits
and puts the result in \texttt{v1}.
 \endtab
\code

void bitvector_rightShift (bitvector_vector *v1, bitvector_vector *v2, int b);
\endcode
 \tab
Performs a right shift of \texttt{v2} by \texttt{b} bits
and puts the result in \texttt{v1}.
 \endtab
\code

void bitvector_leftShiftSelf (bitvector_vector *v, int b);
\endcode
 \tab
Performs a left shift of \texttt{v} by \texttt{b} bits
and puts the result in \texttt{v}.
 \endtab
\code

void bitvector_leftShift1Self (bitvector_vector *v);
\endcode
 \tab
Performs a left shift of \texttt{v} by one bit
and puts the result in \texttt{v}.
 \endtab
\code

void bitvector_rightShiftSelf (bitvector_vector *v, int b);
\endcode
 \tab
Performs a right shift of \texttt{v} by \texttt{b} bits
and puts the result in \texttt{v}.
 \endtab
\code

void bitvector_flip (bitvector_vector *v);
\endcode
 \tab
Flips (or toggle) the values of all the bits of {\tt v} (exchange 0 for 1, and vice-versa).
 \endtab
\code

void bitvector_setMaskLow (bitvector_vector *v, int t);
\endcode
 \tab
Fills {\tt v} with {\tt t} ones followed by zeros.
 \endtab
\code

void bitvector_setInvMaskLow (bitvector_vector *v, int t);
\endcode
 \tab
Fills {\tt v} with {\tt t} zeros followed by ones.
 \endtab
\code

void bitvector_fillRandomBits (bitvector_vector *v);
\endcode
 \tab
Fills the vector {\tt v} with random bits.
 \pierre{This needs to be implemented!  
    For now, it uses rngstreams to fill using blocks of 32 bits. 
		We may prefer to use a 64-bit RNGs.}
 \endtab
\code\hide
#endif
\endhide\endcode

\iffalse  %%%%%%%%%%
{\bf Other popular implementations of bit vectors:}

\mp{Simple tools for bit vectors in C:  \url{http://c-faq.com/misc/bitsets.html}.
Bits are counted from the left, starting at 0, as far as I understand.}

\mp{Another nice set of macros in C:
\url{https://github.com/iplinux/x11proto-trap/blob/master/xtrapbits.h}}

\mp{In \url{https://www.gnu.org/software/guile/docs/docs-1.8/guile-ref/Bit-Vectors.html},
bits are counted from left to right, starting at 0.
Different operations are available, such as Hamming weight of bit string.}

\mp{See \url{https://en.wikipedia.org/wiki/Bit_array} for bit vectors in various languages.
In \texttt{LatNet Builder} we use \texttt{dynamic\_bitset}, available in
the Boost C++ library:
\url{https://www.boost.org/doc/libs/1\_73\_0/libs/dynamic\_bitset/dynamic\_bitset.html}.}

\mp{In the dynamic bitsets of Boost, we  find the following:\small
``Each bit represents either the Boolean value true or false (1 or 0). To set a bit is to assign it 1. To clear or reset a bit is to assign it 0. To flip a bit is to change the value to 1 if it was 0 and to 0 if it was 1. Each bit has a non-negative position. A bitset x contains x.size() bits, with each bit assigned a unique position in the range [0,x.size()). The bit at position 0 is called the least significant bit and the bit at position size() - 1 is the most significant bit. When converting an instance of dynamic\_bitset to or from an unsigned long n, the bit at position i of the bitset has the same value
as {\tt (n >> i) \& 1}.''}
\fi   %%%%%%%%